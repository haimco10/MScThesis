\chapter{Introduction}


Machine learning is a  field of computer science that concerned with data processing and the ability
of the computer to learn from this data. 
One main objective of this field is the development of algorithms capable of inference based on 
observable data, such as text documents, pictures, audio, video etc\ldots 
 
In the \textit{Supervised Learning} setting, the input of the learning algorithm are input-label pairs. 
The goal of the algorithm is to learn the underlying connection between the inputs and their labels, 
thus being able to \textit{predict} a label for a previously unseen input. 
When the possible label values are from a discrete finite set, this learning problem is called
 \textit{classification}. The basic classification problem is the \textit{binary classification}, i.e. classifying 
 each data instance into one of only two possible classes, while in the \textit{multiclass classification} setting, 
 there are three or more possible classes . In contradiction to the multiclass classification task, the binary 
 case is simpler because, eliminating one class, gives you the correct one 
 straightforward. However, this is more difficult in the \textit{multiclass 
 classification}, when even when we eliminate one possible class, we yet have 
 some more classes the we need to decide which of these classes  is the correct one. 
 

\section{Online Learning}
\label{sec:online_learning}

One of the main features of the classification problem is the way that the data is collected. 
In some applications, the labeled data is collected first, so that we have an access to the entire training
 dataset at once. Then, we use this whole examples collection as an input to 
 a \textit{Batch learning} algorithm and learn a classification model about this problem. 
However, in a lot of real life application this is not the case. In some problems like spam filtering, there is
a flow of data that is transmitted in sequence, and it takes time to collect a large amount of data to learn
 from, so we don't want to wait too long before we can have a decent prediction about the continuing 
 incoming examples. For those applications we use  \textit{Online Learning} based algorithms. 
  In this setting, at any time we keep the learned model in memory, and update it when 
  a new labeled example is coming in.
Unlike the \textit{Batch learning}, in the \textit{Online Learning}, 
the learner perception about the classification task become stronger when time is passing by.
 
%Assume you want to buy a house. There are two ways to do that. 
%You can pay all the house cost at once and you get the house, or, 
%if you don't have all the money right now but you do want to get into the house as soon as possible, 
% you can pay a mortgage every month and become closer and closer to a full ownership each month. 
% Collecting a data as  collecting money can be done in two ways. 
% In the \textit{Batch learning} setting we collect a certain amount of data first, and  than, 
% we want to use this data in order to learn how to classify the incoming data instances. 
% However, in a lot of applications there is a flow of data that is transmitted in sequence  
% and we would like to learn on the flow how to classify the data instances. 
% The last method is called \textit{Online Learning}.
%


The \textit{Online Learning} is performed in rounds, where in each round $t$, 
the algorithm gets an input instance $\vxii$ in some domain $\mathcal{X}$, and predicts a  correspond 
measure, $\hat{p_t}$ based on the algorithm decision rule. This measure can be in the label domain, 
$\mathcal{Y}$, or can be mapped into $\hat{y_t}$ which is the predicted label in $\mathcal{Y}$.   
After predicting the label, the true label ($y_t$ in the labels domain, $\mathcal{Y}$)  is revealed 
and the learner suffers a non negative loss of $\l\paren{\hat{p_t},y_t}$ that measures how much the 
prediction is compatible with the true label. The desired property of such function is to generate low 
values when the prediction is close to the actual label in some sense, and high values when the opposite 
is true. Then, the algorithm updates its decision rule, based on the past known data and the revealed label. 

\section{Selective Sampling}
\label{sec:selective_sampling}

Usually, in an  online binary learning task setting, we improve the prediction over time, 
and that means that the algorithm  have less and lees prediction mistakes when it updates its model. 
Sometimes, annotating the data consume expensive resources, like time, money or manpower, 
and we would like to avoid using those resources when we can. In other words, we would like 
to avoid querying labels for the input examples when it is possible. For example, if we  update the 
model only when there is a prediction mistake (as in Perceptron), we actually don't really 
use the information about the correct label when there is no need to update. In such cases, 
it will be helpful to assess every time how much we are sure about our prediction, and no update should be done, 
so no query should be issued, or if we not sure about the prediction, hence we should issue a 
query and update the model using the update rule and the correct label. 
This approach, that queries labels only for selected examples is called \textit{Selective sampling.}

\section{Multi-Task with Shared Annotator}
\label{sec:multi_task_intro}

 In supervised learning setting, the main bottleneck is the need to annotate data. A common protocol is 
 problem centric: first collect data or inputs automatically (with low cost), and then 
 pass it on to a user or an expert to be annotated. Annotation can be outsourced to the crowed by a 
 service like Mechanical Turk (like google's recaptcha project), or performed by experts as
  in the Linguistic data Consortium. Then, this data 
 may be used to build models, either for a single task or many tasks. This approach is not making optimal 
 use of the main resource - the annotator - as some tasks are harder than others, yet we need to give the 
 annotator all the data to be annotated for each task a-priori. 
 
 Another aspect of this problem is the need to adapt systems to individual users, to this end, 
 such systems may query the user for the label of some input. Yet, if a few systems will do so 
 independently, the user will be flooded with queries, and will avoid interaction with those systems. 
 For example, sometimes there is a need to annotate news items from few agencies. One person cannot 
 handle all of them, and only some items can be annotated, which ones? Our setting is designed to handle 
 exactly this problem, and specifically, how to make best usage of annotation time.
 This settings can also handle the case when we want to limit the updates number, 
 for example if we have a lot of clients that generate data, but only one server with a limited computation 
 power is allocated to process the received data and we want to limit the amount of updates for all tasks.
 
 We propose a new framework of online multi-task learning with a shared annotator. 
 Here, algorithms are learning few tasks simultaneously, yet they receive feedback using a central 
 mechanism that trades off the amount of feedback (or labels) each task receives. We derive a specific 
 set of algorithms based on the good-old Perceptron algorithm, called SHAMPO (SHared Annotator for Multiple 
 PrOblems) for binary classification, and analyze it in the mistake bound model, showing that our algorithm 
 may perform well compared with methods that observe all annotated data. we show how to reduce 
 few contextual bandit problems into our framework, and provide specific bounds for such 
settings. We evaluate our algorithm with four different datasets for OCR, vowel prediction (VJ) and 
document classification, and show that it can improve performance either on average over all tasks, 
or even if their output is combined towards a single shared task, such as multi-class prediction.
 We conclude with discussion of related work, and few of the many routes to extend this work.
