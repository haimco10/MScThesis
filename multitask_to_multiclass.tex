\chapter{From Multi-task to Contextual Bandits}\label{chap:multiclass}
Although our algorithm is designed for many binary-classification tasks that can be independent, 
it can also be applied in two settings of contextual bandits, when decoupling exploration and 
exploitation is allowed as in ~\cite{DBLP:conf/icml/YuM09,DBLP:conf/icml/AvnerMS12}. 
The problem of this setting is predicting a label $\hat{Y}_t \in\{ 1 \comdots C\}$ given an input $\vxi{t}$. 
As before, the algorithm works in rounds. On round $t$ the algorithm receives an input $\vxi{t}$ and 
outputs  multicalss label $\hat{Y}_t\in\{1 \comdots C\}$. 
Then, it queries for some information about the label via a single binary ``yes-no'' question, and uses the 
feedback to update its model. We consider here two forms of binary questions.

\section{One-vs-Rest}
The first setting is termed {\em one-vs-rest}. The algorithm asks if the true label is some label 
$\bar{Y}_t\in\{ 1 \comdots C\}$, possibly not the predicted label, i.e. it may be the case that 
$\bar{Y}_t \neq\hat{Y}_t$. Given the response whether  $\bar{Y}_t$ is the true label $Y_t$, the algorithm 
updates its models. The reduction we perform is by introducing $K$ tasks, one per class. 
The problem of the learning algorithm for task $i$ is to decide whether the true label is class $i$ or not. 
Given the output of all the binary classifiers, the algorithm generates a single multi-class prediction to be the 
single label for which the output of the corresponding binary classifier is positive. If such class does not 
exist, or there are more than one classes  as such, a random prediction is used, i.e., given an input $\vxi{t}$ 
we define $\hat{Y}_t = \arg\max_i \hyi{i,t}$, where ties are broken arbitrarily. The label to be queried is 
$\bar{Y}_t=J_t$, i.e. the problem index that SHAMPO is querying. We analyze the performance of this 
reduction as a multiclass prediction algorithm.

\begin{corollary}
Assume the SHAMPO algorithm is executed as above with $K=C$ one-vs-rest problems, on a sequence 
$(\vx_1,Y_1),\ldots,(\vx_n,Y_n)\in\mathbb{R}^d\times\{1,...,C\}$, and input parameter $b>0$  and prior 
$1\le a_i\forall i$. Then for all $\gamma>0$ and all $\vui{i}\in\mathbb{R}^d$, there 
exists $0<\delta\le \sum_{i=1}^{C}a_{i}$ such that the expected number of multi-class errors is bounded as follows
 \[
 \begin{split}
 &\mathbb{E}\brackets{\sum_t \1{Y_t \neq \hat{Y}_t}}\\
 &\le \frac{\delta}{\gamma}\brackets{\left(1+\frac{X^2}{2b} \right){\bar L}_{\gamma,n}+
 \frac{\left({2b+X^2}\right)^2U^2}{8{\gamma}b}}
 +\paren{2\frac{\lambda}{b}-1}\mathbb{E}\brackets{\sum_{i=1}^{K}\sum_{t=1}^{n}{a_{i}G_{i,t}}}~,
 \end{split}
 \]


where $\1{I}=1$ if the predicate $I$ is true, and zero otherwise.
\end{corollary}
The corollary follows directly from \thmref{thm:bound_aggressive_prior} by noting that, 
$\1{Y_t \neq \hat{Y}_t} \leq \sum_i M_{i,t}$. That is, there is a multiclass mistake if there is at least one 
prediction mistake of one of the one-vs-rest problems. The closest setting is contextual bandits, yet we 
allow decoupling of exploration and exploitation. Ignoring this decoupling, the 
Banditron algorithm~\cite{kakade2008efficient} is the closest to ours, 
with a regret of $O(T^{2/3})$. Hazan et al~\cite{hazan2011newtron} proposed an algorithm 
with $O(\sqrt{T})$ regret but designed for the $\log$ loss, with coefficient that may be very large, and 
another~\cite{DBLP:journals/ml/CrammerG13} algorithm has $O(\sqrt{T})$ regret with respect to prediction 
mistakes, yet they assumed stochastic labeling, rather than adversarial.


\section{One-vs-One}
In the second setting, termed by {\em one-vs-one}, the algorithm picks two labels 
$\bar{Y}^+_t,\bar{Y}^-_t \in\{1\dots C\}$, possibly both not the predicted label. 
The feedback for the learner is three-fold: it is $\yi{J_t,t}=+1$ if the first alternative is the correct label, 
$\bar{Y}^+_t=Y_t$, $\yi{J_t,t}=-1$ if the second alternative is the correct label, $\bar{Y}^-_t=Y_t$, 
and it is $\yi{J_t,t}=0$ otherwise (in this case there is no error and we set $M_{J_t,t}=0$). 
The reduction we perform is by introducing $K= {C \choose 2}$ tasks, one per pair of classes.  
The goal of the learning algorithm for a task indexed with two labels $(y_1,y_2)$ is to decide 
which one is the correct label, given it is one of the two. Given the output of all (binary) classifiers the 
algorithm generates a single multi-class prediction using a tournament in a round-robin 
approach~\cite{DBLP:journals/jmlr/Furnkranz02}. If there is no clear winner, a random prediction is used. 
We now analyze the performance of this reduction as a multiclass prediction algorithm.

\begin{corollary}
Assume the SHAMPO algorithm is executed as above, with $K={C \choose 2}$ one-vs-one problems, 
on a sequence $(\vx_1,Y_1),\ldots,(\vx_n,Y_n)\in\mathbb{R}^d\times\{1,\ldots,C\}$, and input parameter 
$b>0$ and prior $1\le a_i\forall i$ . Then for all $\gamma>0$ and all $\vui{i}\in\mathbb{R}^d$, 
there exists $0<\delta\le \sum_{i=1}^{C \choose 2}a_{i}$ such that the expected number of multi-class 
errors can be bounded as follows
 \[
 \begin{split}
 &\mathbb{E}\brackets{\sum_t \1{Y_t \neq \hat{Y}_t}}\le \frac{2}{({C \choose 2}-1)/2+1}\times\\
 &\braces{ \frac{\delta}{\gamma}\brackets{\paren{1+\frac{X^2}{2b} }{\bar L}_{\gamma,n}+\frac{\paren{{2b+X^2}}^2U^{2}}{8{\gamma}b}}+\paren{2\frac{\lambda}{b}-1}\mathbb{E}\brackets{\sum_{i=1}^{K}\sum_{t=1}^{n}{a_{i}G_{i,t}}}}
 \end{split}
 \]

%where $\1{I}=1$ if the predicate $I$ is true, and zero otherwise.
\end{corollary}
The corollary follows directly from \thmref{thm:bound_aggressive_prior} by noting that the  multiclass error 
is bounded by the multitask errors with a constant factor as following, 
\[
\1{Y_t \neq \hat{Y}_t} \leq \frac{2}{({C \choose 2}-1)/2+1}\sum_{i=1}^{ {C \choose 2}} M_{i,t}. 
\]
The last fact is derived from ~\cite{allwein2001reducing} analysis of the training error 
with the corresponding hamming distance of the ECOC matrix for the One-vs-One 
setting. Note, that the bound is essentially  independent of $C$ as the coefficient in the bound is upper 
bounded by $4$.
% as in Allwin et al. in 

We conclude this section with two algorithmic modifications, we employed in this setting. 
Currently, when the feedback is zero, there is no update of the weights, because there are no errors.
 This causes the algorithm to effectively ignore such examples, as in these cases the algorithm is not 
 modifying any model, furthermore, if such example is repeated, a problem with possibly ``0'' feedback 
 may be queried again.
%We fix this issue with one of two modifications: In the first one, if the feedback is zero, we modify the model to reduce the chance that the chosen problem, $J_t$, would be chosen again for the same input (i.e. not to make the same wrong-choice of choosing irrelevant problem again). To this end, we modify the weights a bit, to increase the confidence (absolute margin) of the model for the same input, and replace \eqref{perc_update} with,
\[
\vwi{J_t,t} = \vwi{J_t,t-1}+ \1{\yi{J_t,t}\ne0} \, \yi{J_t,t}\, \vxi{J_t,t} +
 \1{\yi{J_t,t}=0} \eta \hyi{J_t,t}\vxi{J_t,t}~,
\]
for some $\eta>0$. In other words, if there is a possible error (i.e. $\yi{J_t,t}\ne0$) the update follows the 
Perceptron's rule. Otherwise, the weights are updated such that the absolute margin will increase, 
as 
\[
\begin{split}
\vert \vwti{J_t,t} \vxi{J_t,t} \vert &= \paren{\vwi{J_t,t-1}+ \eta \hyi{J_t,t}\vxi{J_t,t})^{\top}\vxi{J_t,t}}\\ 
&= \vert \vwti{J_t,t-1}\vxi{J_t,t}+ \eta \sign(\vwti{J_t,t-1}\vxi{J_t,t}) \Vert\vxi{J_t,t}\Vert^2 \vert\\
&=\vert \vwti{J_t,t-1}\vxi{J_t,t}\vert + \eta \Vert\vxi{J_t,t}\Vert^2 >\vert \vwti{J_t,t-1}\vxi{J_t,t}\vert. 
\end{split}
\]
We call this method {\em one-vs-one-weak}, as it performs weak updates for zero feedback. 
The second alternative is not to allow $0$ value feedback, and if this is the case, to set the label to be 
either $+1$ or $-1$, randomly.
%Both alternates are evaluated below.
We call this method {\em one-vs-one-random}.
