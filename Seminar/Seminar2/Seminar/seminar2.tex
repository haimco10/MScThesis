\documentclass{beamer}

\usepackage{amssymb,amsmath,amsfonts,graphicx}
\usepackage[english]{babel}
\usepackage{graphics}
\usepackage{beamerthemesplit} 
\usepackage{beamerthemeshadow}
\usepackage[latin1]{inputenc}
\usefonttheme{professionalfonts}
\usepackage{times}
\usepackage{amsmath}
\usecolortheme{whale}
\usepackage{enumerate}
\usepackage{amssymb}
\setcounter{tocdepth}{3}
\usepackage{graphicx}
\usepackage{times}
\usepackage{subfigure}
\usepackage{color}
\usepackage{amsbsy}
\usepackage{amsmath}
\usepackage{amsfonts}
\usepackage{amssymb}
\usepackage{epsfig}
\usepackage{tabulary}
\usepackage{wrapfig}
\usepackage[noadjust]{cite}

%%%%%%%%%%%%%%%%%%%%%%%%%%%%%%%%%%%%%%%%%%%%%%%%%%%%%%%%%%%%%%%%%%
% General
%%%%%%%%%%%%%%%%%%%%%%%%%%%%%%%%%%%%%%%%%%%%%%%%%%%%%%%%%%%%%%%%%%
\newfont{\msym}{msbm10}
\newcommand{\reals}{\mathbb{R}}%Re}%{mbox{\msym R}}
\newcommand{\half}{\frac{1}{2}}       
\newcommand{\sign}{{\rm sign}}
\newcommand{\paren}[1]{\left({#1}\right)}
\newcommand{\brackets}[1]{\left[{#1}\right]}
\newcommand{\braces}[1]{\left\{{#1}\right\}}
\newcommand{\ceiling}[1]{\left\lceil{#1}\right\rceil}
\newcommand{\abs}[1]{\left\vert{#1}\right\vert}
\newcommand{\tr}{{\rm Tr}}
\newcommand{\pr}[1]{{\rm Pr}\left[{#1}\right]}
\newcommand{\prp}[2]{{\rm Pr}_{#1}\left[{#2}\right]}
\newcommand{\Exp}[1]{{\rm E}\left[{#1}\right]}
\newcommand{\Expp}[2]{{\rm E}_{#1}\left[{#2}\right]}
\newcommand{\eqdef}{\stackrel{\rm def}{=}}
\newcommand{\comdots}{, \ldots ,}
\newcommand{\true}{\texttt{True}}
\newcommand{\false}{\texttt{False}}
\newcommand{\mcal}[1]{{\mathcal{#1}}}
\newcommand{\argmin}[1]{\underset{#1}{\mathrm{argmin}} \:}
\newcommand{\normt}[1]{\left\Vert {#1} \right\Vert^2}
\newcommand{\step}[1]{\left[#1\right]_+}
\newcommand{\1}[1]{[\![{#1}]\!]}
\newcommand{\diag}{{\textrm{diag}}}
%\newcommand{\det}{{\textrm{det}}}
\newcommand{\KL}{{\textrm{D}_{\textrm{KL}}}}
\newcommand{\IS}{{\textrm{D}_{\textrm{IS}}}}
\newcommand{\EU}{{\textrm{D}_{\textrm{EU}}}}
\newcommand{\lloss}{\ell} % label loss
\newcommand{\closs}{\hat\ell} % classifier loss
\newcommand{\st}{\textrm{s.t.}} % such that / subject to

%%%%%%%%%%%%%%%%%%%%%%%%%%%%%%%%%%%%%%%%%%%%%%%%%%%%%%%%%%
% Control symbols
%%%%%%%%%%%%%%%%%%%%%%%%%%%%%%%%%%%%%%%%%%%%%%%%%%%%%%%%%%
\newcommand{\leftmarginpar}[1]{\marginpar[#1]{}}
\newcommand{\figline}{\rule{0.50\textwidth}{0.5pt}}
\newcommand{\pseudocodefont}{\normalsize}
\newcommand{\nolineskips}{
\setlength{\parskip}{0pt}
\setlength{\parsep}{0pt}
\setlength{\topsep}{0pt}
\setlength{\partopsep}{0pt}
\setlength{\itemsep}{0pt}}

%%%%%%%%%%%%%%%%%%%%%%%%%%%%%%%%%%%%%%%%%%%%%%%%%%%%%%%%%%%
% Equations and references
%%%%%%%%%%%%%%%%%%%%%%%%%%%%%%%%%%%%%%%%%%%%%%%%%%%%%%%%%%%
\newcommand{\beq}[1]{\begin{equation}\label{#1}}
\newcommand{\eeq}{\end{equation}}
\newcommand{\beqa}{\begin{eqnarray}}
\newcommand{\eeqa}{\end{eqnarray}}
%\renewcommand{\eqref}[1]{Eq.~(\ref{#1})}
\newcommand{\exmref}[1]{Example~\ref{#1}} 
\newcommand{\sthmref}[1]{Thm.~\ref{#1}}  
\newcommand{\remref}[1]{Remark~\ref{#1}} 
\newcommand{\claimref}[1]{Claim~\ref{#1}} 
\newcommand{\corref}[1]{Corollary~\ref{#1}} 
\newcommand{\scorref}[1]{Cor.~\ref{#1}} 
\newcommand{\tran}[1]{{#1}^{\top}}
\newcommand{\norm}{\mcal{N}}
\newcommand{\eqsref}[1]{Eqns.~(\ref{#1})}

% Alex's macros

\newcommand{\chaplabel}[1]{\label{chap:#1}}
\newcommand{\chapref}[1]{Chapter~\ref{chap:#1}}

\newcommand{\seclabel}[1]{\label{sec:#1}}
\newcommand{\secref}[1]{Section~\ref{sec:#1}}

\newcommand{\applabel}[1]{\label{app:#1}}
\newcommand{\appref}[1]{Appendix~\ref{app:#1}} 

\newcommand{\figlabel}[1]{\label{fig:#1}}
\newcommand{\figref}[1]{Figure~\ref{fig:#1}}

\newcommand{\tablabel}[1]{\label{tab:#1}}
\newcommand{\tabref}[1]{Table~\ref{tab:#1}}

\newcommand{\eqlabel}[1]{\label{eq:#1}}
\renewcommand{\eqref}[1]{Equation~(\ref{eq:#1})}

\newcommand{\proplabel}[1]{\label{prop:#1}}
\newcommand{\propref}[1]{Proposition~\ref{prop:#1}}

\newcommand{\deflabel}[1]{\label{def:#1}}
\newcommand{\defref}[1]{Definition~\ref{def:#1}}

\newcommand{\lemlabel}[1]{\label{lem:#1}}
\newcommand{\lemref}[1]{Lemma~\ref{lem:#1}}

\newcommand{\thmlabel}[1]{\label{thm:#1}}
\newcommand{\thmref}[1]{Theorem~\ref{thm:#1}}

\newcommand{\alglabel}[1]{\label{alg:#1}}
\newcommand{\algref}[1]{Algorithm~\ref{alg:#1}}


%%%%%%%%%%%%%%%%%%%%%%%%%%%%%%%%%%%%%%%%%%%%%%%%%%%%%%%%%%%
% bold, up, down
%%%%%%%%%%%%%%%%%%%%%%%%%%%%%%%%%%%%%%%%%%%%%%%%%%%%%%%%%%%
\newcommand{\mb}[1]{{\boldsymbol{#1}}}
\newcommand{\up}[2]{{#1}^{#2}}
\newcommand{\dn}[2]{{#1}_{#2}}
\newcommand{\du}[3]{{#1}_{#2}^{#3}}
\renewcommand{\star}[1]{\up{#1}{*}}
\newcommand{\textl}[2]{{$\textrm{#1}_{\textrm{#2}}$}}


%%%%%%%%%%%%%%%%%%%%%%%%%%%%%%%%%%%%%%%%%%%%%%%%%%%%%%%%%%%
% vectors \va
%%%%%%%%%%%%%%%%%%%%%%%%%%%%%%%%%%%%%%%%%%%%%%%%%%%%%%%%%%%
\newcommand{\vx}{\mb{x}} 
\newcommand{\vxi}[1]{\vx_{#1}}
\newcommand{\vxii}{\vxi{i}}

\newcommand{\yi}[1]{y_{#1}}
\newcommand{\yii}{\yi{i}}
\newcommand{\hy}{\hat{y}}
\newcommand{\hyi}[1]{\hat{y}_{#1}}
\newcommand{\hyii}{\hyi{i}}

\newcommand{\vy}{\mb{y}} 
\newcommand{\vyi}[1]{\vy_{#1}}
\newcommand{\vyii}{\vyi{i}}

\newcommand{\vn}{\mb{\nu}} 
\newcommand{\vni}[1]{\vn_{#1}}
\newcommand{\vnii}{\vni{i}}

\newcommand{\vmu}{\mb{\mu}}
\newcommand{\vmus}{{\vmu^*}}
\newcommand{\vmuts}{{\vmus}^{\top}}
\newcommand{\vmui}[1]{\vmu_{#1}}
\newcommand{\vmuii}{\vmui{i}}

\newcommand{\vmut}{\vmu^{\top}}
\newcommand{\vmuti}[1]{\vmut_{#1}}
\newcommand{\vmutii}{\vmuti{i}}

\newcommand{\vsigma}{\mb \sigma}
\newcommand{\msigma}{\Sigma}
\newcommand{\msigmas}{{\msigma^*}}
\newcommand{\msigmai}[1]{\msigma_{#1}}
\newcommand{\msigmaii}{\msigmai{i}}

\newcommand{\mups}{\Upsilon}
\newcommand{\mupss}{{\mups^*}}
\newcommand{\mupsi}[1]{\mups_{#1}}
\newcommand{\mupsii}{\mupsi{i}}
\newcommand{\upssl}{\upsilon^*_l}


\newcommand{\vu}{\mb{u}} 
\newcommand{\vut}{\tran{\vu}}
\newcommand{\vui}[1]{\vu_{#1}}
\newcommand{\vuti}[1]{\vut_{#1}}
\newcommand{\hvu}{\hat{\vu}}
\newcommand{\hvut}{\tran{\hvu}}
\newcommand{\hvur}[1]{\hvu_{#1}}
\newcommand{\hvutr}[1]{\hvut_{#1}}
\newcommand{\vw}{\mb{w}} 
\newcommand{\vwi}[1]{\vw_{#1}}
\newcommand{\vwii}{\vwi{i}}

\newcommand{\vwt}{\tran{\vw}}
\newcommand{\vwti}[1]{\vwt_{#1}}
\newcommand{\vwtii}{\vwti{i}}

\newcommand{\vv}{\mb{v}} 
\newcommand{\vvt}{\tran{\vv}}

\newcommand{\vvi}[1]{\vv_{#1}}
\newcommand{\vvti}[1]{\vvt_{#1}}
\newcommand{\lambdai}[1]{\lambda_{#1}}
\newcommand{\Lambdai}[1]{\Lambda_{#1}}

\newcommand{\vxt}{\tran{\vx}}
\newcommand{\hvx}{\hat{\vx}}
\newcommand{\hvxi}[1]{\hvx_{#1}}
\newcommand{\hvxii}{\hvxi{i}}
\newcommand{\hvxt}{\tran{\hvx}}
\newcommand{\hvxti}[1]{\hvxt_{#1}}
\newcommand{\hvxtii}{\hvxti{i}}
\newcommand{\vxti}[1]{\vxt_{#1}}
\newcommand{\vxtii}{\vxti{i}}

%%%%%%%%%%%%%%%%%%%%%%%%%%%%%%%%%%%%%%%%%%%%%%%%%%%%%%%%%%%%%%%%%
% Matrices (\mA)
%%%%%%%%%%%%%%%%%%%%%%%%%%%%%%%%%%%%%%%%%%%%%%%%%%%%%%%%%%%%%%%%%


\renewcommand{\mp}{P}
\newcommand{\mpd}{\mp^{(d)}}
\newcommand{\mpt}{\mp^T}
\newcommand{\tmp}{\tilde{\mp}}
\newcommand{\mpi}[1]{\mp_{#1}}
\newcommand{\mpti}[1]{\mpt_{#1}}
\newcommand{\mptii}{\mpti{i}}
\newcommand{\mpii}{\mpi{i}}
\newcommand{\mps}{Q}
\newcommand{\mpsi}[1]{\mps_{#1}}
\newcommand{\mpsii}{\mpsi{i}}
\newcommand{\tmpt}{\tmp^T}
\newcommand{\mz}{Z}
\newcommand{\mv}{V}
\newcommand{\mvi}[1]{\mv_{#1}}
\newcommand{\mvt}{V^T}
\newcommand{\mvti}[1]{\mvt_{#1}}
\newcommand{\mzt}{\mz^T}
\newcommand{\tmz}{\tilde{\mz}}
\newcommand{\tmzt}{\tmz^T}
\newcommand{\mx}{\mathbf{X}}
\newcommand{\ma}{\mathbf{A}}
\newcommand{\mxs}[1]{\mx_{#1}}


\newcommand{\mxi}[1]{\textrm{diag}^2\paren{\vxi{#1}}}
\newcommand{\mxii}{\mxi{i}}

%\newcommand{\mxi}[1]{\mx_{#1}}
%\newcommand{\mxii}{\mxi{i}}
\newcommand{\hmx}{\hat{\mx}}
\newcommand{\hmxi}[1]{\hmx_{#1}}
\newcommand{\hmxii}{\hmxi{i}}
\newcommand{\hmxt}{\hmx^T}
\newcommand{\mxt}{\mx^\top}
\newcommand{\mi}{I}
\newcommand{\mq}{Q}
\newcommand{\mqt}{\mq^T}
\newcommand{\mlam}{\Lambda}
%\newcommand{\ma}{A}
%\newcommand{\ms}{S}
%\newcommand{\mt}{T}

%%%%%%%%%%%%%%%%%%%%%%%%%%%%%%%%%%%%%%%%%%%%%%%%%%%%%%%%%%%
% mathcal 
%%%%%%%%%%%%%%%%%%%%%%%%%%%%%%%%%%%%%%%%%%%%%%%%%%%%%%%%%%%
\renewcommand{\L}{\mcal{L}}
\newcommand{\R}{\mcal{R}}
\newcommand{\X}{\mcal{X}}
\newcommand{\Y}{\mcal{Y}}
\newcommand{\F}{\mcal{F}}
\newcommand{\nur}[1]{\nu_{#1}}
\newcommand{\lambdar}[1]{\lambda_{#1}}
\newcommand{\gammai}[1]{\gamma_{#1}}
\newcommand{\gammaii}{\gammai{i}}
\newcommand{\alphai}[1]{\alpha_{#1}}
\newcommand{\alphaii}{\alphai{i}}
\newcommand{\lossp}[1]{\ell_{#1}}
\newcommand{\eps}{\epsilon}
\newcommand{\epss}{\eps^*}
\newcommand{\lsep}{\lossp{\eps}}
\newcommand{\lseps}{\lossp{\epss}}
\newcommand{\T}{\mcal{T}}

%%%%%%%%%%%%%%%%%%%%%%%%%%%%%%%%%%%%%%%%%%%%%%%%%%%%%%%%%%%
% Notes
%%%%%%%%%%%%%%%%%%%%%%%%%%%%%%%%%%%%%%%%%%%%%%%%%%%%%%%%%%%
\newcommand{\kc}[1]{\begin{center}\fbox{\parbox{3in}{{\textcolor{green}{KC: #1}}}}\end{center}}
\newcommand{\fp}[1]{\begin{center}\fbox{\parbox{3in}{{\textcolor{red}{FP: #1}}}}\end{center}}
\newcommand{\md}[1]{\begin{center}\fbox{\parbox{3in}{{\textcolor{blue}{MD: #1}}}}\end{center}}
\newcommand{\ak}[1]{\begin{center}\fbox{\parbox{3in}{{\textcolor{yellow}{AK: #1}}}}\end{center}}




\newcommand{\newstuffa}[2]{#2}
\newcommand{\newstufffroma}[1]{}
\newcommand{\newstufftoa}{}
%\newcommand{\newstuffa}[2]{~\\{\color{MyRed} #1:\\ }{\textcolor{MyGray}{#2}~\\}}
%\newcommand{\newstufffroma}[1]{~\\{\color{MyRed} #1:\\ }\color{MyGray}}
%\newcommand{\newstufftoa}{\color{black}}

\newcommand{\newstuff}[2]{#2}
\newcommand{\newstufffrom}[1]{}
\newcommand{\newstuffto}{}
\newcommand{\oldnote}[2]{}

%%%%\newcommand{\comment}[1]{}
\newcommand{\commentout}[1]{}
\newcommand{\mypar}[1]{\medskip\noindent{\bf #1}}


%%%%%%%%%%%%%%%%%%%%%%%%%%%%%%%%%%%%%%%%%%%%%%%%%%%%%%%%%%%
% other
%%%%%%%%%%%%%%%%%%%%%%%%%%%%%%%%%%%%%%%%%%%%%%%%%%%%%%%%%%%
% inner products
\newcommand{\inner}[2]{\left< {#1} , {#2} \right>}
\newcommand{\kernel}[2]{K\left({#1},{#2} \right)}
\newcommand{\tprr}{\tilde{p}_{rr}}
\newcommand{\hxr}{\hat{x}_{r}}
\newcommand{\projalg}{{PST }}%{\tt Projection }}
\newcommand{\projealg}[1]{$\textrm{PST}_{#1}~$}%{\tt Projection }}
\newcommand{\gradalg}{{GST }}%\tt Gradient }}



\newcounter {mySubCounter}
\newcommand {\twocoleqn}[4]{
  \setcounter {mySubCounter}{0} %
  \let\OldTheEquation \theequation %
  \renewcommand {\theequation }{\OldTheEquation \alph {mySubCounter}}%
  \noindent \hfill%
  \begin{minipage}{.40\textwidth}
\vspace{-0.6cm}
    \begin{equation}\refstepcounter{mySubCounter}
      #1 
    \end {equation}
  \end {minipage}
~~~~~~
%\hfill %
  \addtocounter {equation}{ -1}%
  \begin{minipage}{.40\textwidth}
\vspace{-0.6cm}
    \begin{equation}\refstepcounter{mySubCounter}
      #3 
    \end{equation}
  \end{minipage}%
  \let\theequation\OldTheEquation
}


\newcommand{\vzero}{\mb{0}} 

\newcommand{\smargin}{\mcal{M}}

\newcommand{\ai}[1]{A_{#1}}
\newcommand{\bi}[1]{B_{#1}}
\newcommand{\aii}{\ai{i}}
\newcommand{\bii}{\bi{i}}
\newcommand{\betai}[1]{\beta_{#1}}
\newcommand{\betaii}{\betai{i}}
\newcommand{\mar}{M}
\newcommand{\mari}[1]{\mar_{#1}}
\newcommand{\marii}{\mari{i}}
\newcommand{\nmari}[1]{m_{#1}}
\newcommand{\nmarii}{\nmari{i}}


%\newcommand{\erf}{\mathrm{erf}}
\newcommand{\erf}{\Phi}


\newcommand{\var}{V}
\newcommand{\vari}[1]{\var_{#1}}
\newcommand{\varii}{\vari{i}}

\newcommand{\varb}{v}
\newcommand{\varbi}[1]{\varb_{#1}}
\newcommand{\varbii}{\varbi{i}}

%\newcommand{\vara}{v^+}
\newcommand{\vara}{u}
\newcommand{\varai}[1]{\vara_{#1}}
\newcommand{\varaii}{\varai{i}}

\newcommand{\marb}{m}
\newcommand{\marbi}[1]{\marb_{#1}}
\newcommand{\marbii}{\marbi{i}}

\newcommand{\algname}{{AROW}}
\newcommand{\rlsname}{{RLS}}
\newcommand{\mrlsname}{{MRLS}}


%\newcommand{phi1}{{1+\frac{\phi}{2}}}
\newcommand{\phia}{\psi}
\newcommand{\phib}{\xi}


\newcommand{\amsigmaii}{\tilde{\msigma}_i}
\newcommand{\amsigmai}[1]{\tilde{\msigma}_{#1}}
\newcommand{\avmuii}{\tilde{\vmu}_i}
\newcommand{\avmui}[1]{\tilde{\vmu}_{#1}}
\newcommand{\amarbii}{\tilde{\marb}_i}
\newcommand{\avarbii}{\tilde{\varb}_i}
\newcommand{\avaraii}{\tilde{\vara}_i} 
\newcommand{\aalphaii}{\tilde{\alpha}_i}

\newcommand{\svar}{v}
\newcommand{\smar}{m}
\newcommand{\nsmar}{\bar{m}}

\newcommand{\vnu}{\mb{\nu}}
\newcommand{\vnut}{\vnu^\top}
\newcommand{\vz}{\mb{z}} 
\newcommand{\vZ}{\mb{Z}}
\newcommand{\fphi}{f_{\phi}}
\newcommand{\gphi}{g_{\phi}}

%%% Local Variables: 
%%% mode: latex
%%% TeX-master: "nips2007"
%%% End: 


\newcommand{\vtmui}[1]{\tilde{\vmu}_{#1}}
\newcommand{\vtmuii}{\vtmui{i}}


\newcommand{\zetai}[1]{\zeta_{#1}}
\newcommand{\zetaii}{\zetai{i}}



%%%%%%

\newcommand{\vstate}{\bf{s}}
\newcommand{\vstatet}[1]{\vstate_{#1}}
\newcommand{\vstatett}{\vstatet{t}}

\newcommand{\mtran}{\bf{\Phi}}
\newcommand{\mtrant}[1]{\mtran_{#1}}
\newcommand{\mtrantt}{\mtrant{t}}

\newcommand{\vstatenoise}{\bf{\eta}}
\newcommand{\vstatenoiset}[1]{\vstatenoise_{#1}}
\newcommand{\vstatenoisett}{\vstatenoiset{t}}


\newcommand{\vobser}{\bf{o}}
\newcommand{\vobsert}[1]{\vobser_{#1}}
\newcommand{\vobsertt}{\vobsert{t}}

\newcommand{\mobser}{\bf{H}}
\newcommand{\mobsert}[1]{\mobser_{#1}}
\newcommand{\mobsertt}{\mobsert{t}}

\newcommand{\vobsernoise}{\bf{\nu}}
\newcommand{\vobsernoiset}[1]{\vobsernoise_{#1}}
\newcommand{\vobsernoisett}{\vobsernoiset{t}}

\newcommand{\mstatenoisecov}{\bf{Q}}
\newcommand{\mstatenoisecovt}[1]{\mstatenoisecov_{#1}}
\newcommand{\mstatenoisecovtt}{\mstatenoisecovt{t}}

\newcommand{\mobsernoisecov}{\bf{R}}
\newcommand{\mobsernoisecovt}[1]{\mobsernoisecov_{#1}}
\newcommand{\mobsernoisecovtt}{\mobsernoisecovt{t}}



\newcommand{\vestate}{\bf{\hat{s}}}
\newcommand{\vestatet}[1]{\vestate_{#1}}
\newcommand{\vestatett}{\vestatet{t}}
\newcommand{\vestatept}[1]{\vestatet{#1}^+}
\newcommand{\vestatent}[1]{\vestatet{#1}^-}


\newcommand{\mcovar}{\bf{P}}
\newcommand{\mcovart}[1]{\mcovar_{#1}}
\newcommand{\mcovarpt}[1]{\mcovart{#1}^+}
\newcommand{\mcovarnt}[1]{\mcovart{#1}^-}

\newcommand{\mkalmangain}{\bf{K}}
\newcommand{\mkalmangaint}[1]{\mkalmangain_{#1}}


\newcommand{\vkalmangain}{\bf{\kappa}}
\newcommand{\vkalmangaint}[1]{\vkalmangain_{#1}}



\newcommand{\obsernoise}{{\nu}}
\newcommand{\obsernoiset}[1]{\obsernoise_{#1}}
\newcommand{\obsernoisett}{\obsernoiset{t}}

\newcommand{\obsernoisecov}{r}
\newcommand{\obsernoisecovt}[1]{\obsernoisecov_{#1}}
\newcommand{\obsernoisecovtt}{\obsernoisecov}%t{t}}


\newcommand{\obsnscv}{s}
\newcommand{\obsnscvt}[1]{\obsnscv_{#1}}
\newcommand{\obsnscvtt}{\obsnscvt{t}}


\newcommand{\Psit}[1]{\Psi_{#1}}
\newcommand{\Psitt}{\Psit{t}}

\newcommand{\Omegat}[1]{\Omega_{#1}}
\newcommand{\Omegatt}{\Omegat{t}}


\newcommand{\ellt}[1]{\ell_{#1}}
\newcommand{\gllt}[1]{g_{#1}}

\newcommand{\chit}[1]{\chi_{#1}}

\newcommand{\ms}{\mathcal{M}}
\newcommand{\us}{\mathcal{U}}
\newcommand{\as}{\mathcal{A}}

\newcommand{\mn}{M}
\newcommand{\un}{U}

\newcommand{\set}{S}
\newcommand{\seti}[1]{S_{#1}}

\newcommand{\obj}{\mcal{C}}





\usetheme{Frankfurt}


\title{Learning Drifting Data \\Using Selective Sampling}    % Enter 


\begin{document}
\maketitle
\section{Introduction}


\section{Classification Analysis}
\begin{frame}{Objectives}
Approaching the problem of shifting concept in an on-line learning classification setting we set the following objectives:
\begin{enumerate}
\item Detect the switch
\item If switch is undetected - assure that the additional regret it causes is small
\item No false detections
\end{enumerate}

\end{frame}

\begin{frame}{Problem Setting}
We work under the following assumptions:
\begin{itemize}
\item $y_t\in\{\pm1\}$
\item $\vxi{t}\in R^d$
\item for $t\leq\tau$ holds $\Exp{y_t}=\vut\vxi{t}$
\item for $t>\tau$ holds $\Exp{y_t}=\vvt\vxi{t}$
\item $\|\vxi{t}\|=\|\vu\|=\|\vv\|=1$
\end{itemize}


\end{frame}

\begin{frame}{BBQ Algorithm}
We submit prediction:
\begin{equation}
\hat{y}_t=\sign\left\{\vwti{t}\vxi{t}\right\}
\end{equation}
\newline
$\vwi{t}$ is our estimation to the optimal linear classifier obtained by solving the following problem:
\begin{equation}
\vwi{t}=\min_{\vw\in R^d}{\left\{\sum\limits_{i=1}^{n}\left(y_i-\vwt\vxi{i}\right)^2+\|\vw\|^2\right\}}
\label{RLS_prob}
\end{equation}
with $n=N_t$ being the number of queries issued until round $t-1$
\end{frame}

\begin{frame}{BBQ Algorithm}
The solution to equation \ref{RLS_prob} is:
\begin{equation}
\vwi{t}=\left(I+S_{t-1}S_{t-1}^T+ \vxi{t} \vxti{t}\right)^{-1}S_{t-1}Y_{t-1}
\label{RLS}
\end{equation}
where $S_{t-1}=(\vxi{1},...,\vxi{n})\in R^{d\times n}$ and $Y_{t-1}=(y_1,...,y_n)\in R^n$.\newline
Another formulation:
\begin{equation}
\vwi{t}=A_{t}^{-1}b_t
\label{RLS2}
\end{equation}
where $A_t=I+\sum\limits_{i=1}^{n}\vxi{i} \vxti{i}+ \vxi{t} \vxti{t}$ and $b_t=\sum\limits_{i=1}^{n}y_i\vxi{i}$ 
\end{frame}

\begin{frame}{BBQ Algorithm - Querying Labels}
We define:
\begin{equation}
r_t=\vxti{t}A_{t}^{-1}\vxi{t}
\end{equation}
\newline
\newline
A query will be issued at round $t$ if $r_t> t^{-\kappa}$.\newline
\newline
\newline
If $r_t\leq t^{-\kappa}$ the value of the label $y_t$ will remain unknown.
\end{frame}

\begin{frame}{Effect of Switch on BBQ Algorithm}
In the normal setting the BBQ algorithm works well - with logarithmic regret:
\begin{equation}
 R_T\leq O\left(d\ln{T}\right)
\end{equation}
while maintaining significantly reduced amount of quired labels:
\begin{equation}
N_T\sim dT^{\kappa}\ln{T}
\end{equation}
However as switch of the optimal classifier from $\vu$ to $\vv$ at round $\tau$ increases regret bound:
\begin{equation}
 R_T\leq O\left(\|\vv-\vu\|^2{\tau}^{2\kappa}\left(d\ln{\tau}\right)^2d\ln{T}\right)
\end{equation}
 
\end{frame}


\begin{frame}{Effect of Switch on BBQ Algorithm}
The increase in the regret bound is due to increase in the bound of the classifier's bias, after the switch:
\begin{equation}
B_t=\vwti{t}\vxi{t}-\Exp{\vwti{t}\vxi{t}}\leq r_t+\sqrt{r_t}+N_{\tau}\|\vv-\vu\|\sqrt{r_t}
\end{equation}
Instead of 
\begin{equation}
B_t=\vwti{t}\vxi{t}-\Exp{\vwti{t}\vxi{t}}\leq r_t+\sqrt{r_t}
\end{equation}
prior to the switch
\end{frame}

\begin{frame}{Using Selective Sampling to Overcome Switch}
Selective sampling concept gives us confidence on our prediction.\newline
\newline
The term $r_t$ controls and both the bias from the optimal classifier and the instantaneous regret:
\begin{itemize}
\item If $r_t$ is large, then in any case, switch or none, we can not assure low regret.
\item If $r_t$ is small, we should suffer low regret - meaning our prediction should be close enough to the optimal prediction. Unless a switch had occurred...
\end{itemize}
\end{frame}

\begin{frame}{Using Selective Sampling to Overcome Switch}
Main idea - use instances with small $r_t$ to detect switch. An "error" on such instances will be improbable and if it does occur- it must be due to a switch.\newline
\newline
But what is an "error"? - even if we know the optimal classifier $\vu$ the probability for a classification error is $\frac{1-\left|\vut\vxi{t}\right|}{2}$. So error can only be considered in terms of distance from the optimal classifier.
\newline
\newline
Problem - the optimal classifier is unknown. So how can we check if our prediction is close enough to it?

\end{frame}

\begin{frame}{Using Selective Sampling to Overcome Switch}
Solution - estimate optimal classifier $\vv$ with a demo classifier $h_t$ constructed from a window of recent instances.\newline
\begin{itemize}
\item If no switch occurred - $\vxi{t}$ and $h_t$ should give close predictions, as both are close in prediction to $\vv$.
\newline
\item If a switch occurred:
\begin{itemize}
\item If $\vxi{t}$ and $h_t$ do not yield close predictions - we detect the switch
\item If $\vxi{t}$ and $h_t$ t yield close predictions -  switch is insignificant and not much additional regret will be suffered
\end{itemize}
\end{itemize}
\end{frame}

\begin{frame}{Construction of Demo Classifier}
\begin{itemize}
\item Set $L_t=L_0+\sqrt{t}$
\item At round $t$  select a window of last $L_t$ instances 
\item Calculate $A_{L_t}=I+\sum\limits_{l=t-L}^{t-1}{\vxi{l}\vxti{l}}, b_{L_t}=\sum\limits_{l=t-L}^{t}y_l\vxi{l}$
\item Construct $h_t=\left(A_{L_t}+\vxi{t}\vxti{t}\right)^{-1}b_{L_t}$
\end{itemize}
To save querying labels we set resolution classifier $h_t$ for a window of $KL_t$ next instances. At round $KL_t+1$ we construct a new demo classifier, and so forth.
\end{frame}

\begin{frame}{Algorithm for Detecting Switch}
\begin{itemize}
\item Set $\delta_t=\frac{\delta}{t(t+1)}$
\item Calculate $C_t=\left\vert\vwti{t}\vxi{t}-h_{t}^{\top}\vxi{t}\right\vert$
\item Calculate: \newline \newline $K_t=\sqrt{2r_t\ln{\frac{2}{\delta_t}}}+\sqrt{2r_{L_t}\ln{\frac{2}{\delta_t}}}+r_t+\sqrt{r_t}+r_{L_t}+\sqrt{r_{L_t}}$
\item If $C_t>K_t$ declare switch and restart classifier $w_t$ from zero
\item Else continue to next round
\end{itemize}
\end{frame}


\begin{frame}{Algorithm for Detecting Switch}
\begin{itemize}
\item If $C_t>K_t$ switch is detected and we overcome its effect
\item If no switch occurred we can assure that $C_t\leq K_t$ and no false detections will be made
\item If $C_t\leq K_t$ but a switch did occur - can we assure that it will cause no significant additional regret?
\end{itemize}
First we will show that indeed if $C_t\leq K_t$ we can assure low regret.\newline
Later we will prove that the probability for a false positive is small.
\end{frame}

\begin{frame}{Regret Calculation}
The instantaneous regret is controlled by the term $|\vwti{t}\vxi{t}-\vvt\vxi{t}|$:
\begin{eqnarray}
&&R_t=\pr{y_t\vwti{t}\vxi{t}<0}-\pr{y_t\vvt\vxi{t}<0}\leq\nonumber\\
&&\varepsilon I_{\{\left|\vvt\vxi{t}\right|<\varepsilon\}}+\pr{\left|\vwti{t}\vxi{t}-\vvt\vxi{t}\right|\geq\varepsilon}
\label{regret_wt_ht1}
\end{eqnarray}
We can bound it by triangle inequality:
\begin{eqnarray}
&&\left\vert\vwti{t}\vxi{t}-\vvt\vxi{t}\right\vert\leq\left\vert\vwti{t}\vxi{t}-h_{t}^{\top}\vxi{t}\right\vert+\left\vert\vvti{t}\vxi{t}-h_{t}^{\top}\vxi{t}\right\vert\nonumber\\
&&=C_t+\left\vert\vvti{t}\vxi{t}-h_{t}^{\top}\vxi{t}\right\vert
\label{triangl_ht_wt}
\end{eqnarray}
\end{frame}

\begin{frame}{Regret Calculation}
We already have a bound for $C_t$, as a switch was not detected. What about $\left\vert\vvti{t}\vxi{t}-h_{t}^{\top}\vxi{t}\right\vert$?\newline\newline
From the bias bound on the BBQ classifier and by Hoefding bound we shall have:
\begin{equation}
\left\vert\vvt\vxi{t}-h_{t}^{\top}\vxi{t}\right\vert\leq\sqrt{2r_{L_t}\ln{\frac{2}{\delta_t}}}+r_{L_t}+\sqrt{r_{L_t}}
\label{rLt_false}
\end{equation}
With probability $1-\delta_t$.
\end{frame}



\begin{frame}{Regret Calculation}
Combining given bound on $C_t$ and equation \ref{rLt_false} we have:
\begin{eqnarray}
&&\left\vert\vwti{t}\vxi{t}-\vvt\vxi{t}\right\vert\leq\sqrt{r_t}\left(\sqrt{2\ln{\frac{2}{\delta_t}}}+1\right)+r_t\nonumber\\
&&+2\sqrt{r_{L_t}}\left(\sqrt{2\ln{\frac{2}{\delta_t}}}+1\right)+2r_{L_t}
\label{triangl_ht_wt2}
\end{eqnarray}
Equation \ref{triangl_ht_wt2} together with the identity  $I_{\{x<1\}}\leq e^{1-x}$ will allow us to bound the regret.
\end{frame}


\begin{frame}{Regret Calculation}
\begin{eqnarray}
&&\pr{\left|\vwti{t}\vxi{t}-\vvt\vxi{t}\right|\geq\varepsilon}\leq 2I_{\big\{r_{L_t}\left(\sqrt{2\ln{\frac{2}{\delta_t}}}+1\right)^2\geq\frac{\varepsilon^2}{36}\big\}}\\
&&+2I_{\big\{r_{L_t}\geq\frac{\varepsilon}{6}\big\}}+I_{\big\{r_t\left(\sqrt{2\ln{\frac{2}{\delta_t}}}+1\right)^2\geq\frac{\varepsilon^2}{36}\big\}}+I_{\big\{r_t\geq\frac{\varepsilon}{6}\big\}}\nonumber\\
&&\leq2\exp\left\{1-\frac{\varepsilon^2}{36r_{L_t}\left(\sqrt{2\ln{\frac{2}{\delta_t}}}+1\right)^2}\right\}+2\exp\left\{1-\frac{\varepsilon}{6r_{L_t}}\right\}\nonumber\\
&&+\exp\left\{1-\frac{\varepsilon^2}{36r_t\left(\sqrt{2\ln{\frac{2}{\delta_t}}}+1\right)^2}\right\}+\exp\left\{1-\frac{\varepsilon}{6r_t}\right\}\nonumber
\label{exponent_regret_class}
\end{eqnarray}
\end{frame}

\begin{frame}{Regret Calculation}
The cumulative regret is given by:
\begin{equation}
R_T=\sum\limits_{t=1}^{T}R_t
\label{cum_reg_define}
\end{equation}
We will sum over the terms of equation \ref{exponent_regret_class} to bound the regret.\newline\newline
We divide the summation to rounds where $r_t\leq t^{-\kappa}$ and rounds where $r_t\leq t^{-\kappa}$.
\end{frame}

\begin{frame}{Regret Calculation}
We use the identities:  $1-x\leq -\ln{x}$ (for $x\leq1$) and $\exp\{-x\}\leq\frac{1}{ex}$, and the fact that:
\begin{equation}
 r_t\leq 1-\frac{\det{A_{t-1}}}{\det{A_t}}
\end{equation}
to calculate the following sum:
\begin{eqnarray}
&&\sum\limits_{t=T_1,r_t> t^{-\kappa}}^{T}\exp\left\{1-\frac{\varepsilon^2}{36r_t\left(\sqrt{2\ln{\frac{2}{\delta_t}}}+1\right)^2}\right\}\leq\nonumber
\end{eqnarray}
\end{frame}
\begin{frame}{Regret Calculation}

\begin{eqnarray}
&&\leq\frac{36\left(\sqrt{2\ln{\frac{2}{\delta_T}}}+1\right)^2}{\varepsilon^2}\sum\limits_{t=T_1,r_t>t^{-\kappa}}^{T}r_t\nonumber\\
&&\leq\frac{36\left(\sqrt{2\ln{\frac{2}{\delta_T}}}+1\right)^2}{\varepsilon^2}\sum\limits_{t=T_1,r_t>t^{-\kappa}}^{T}\left(1-\frac{\det{A_{t-1}}}{\det{A_t}}\right)\nonumber\\
&&\leq -\frac{36\left(\sqrt{2\ln{\frac{2}{\delta_T}}}+1\right)^2}{\varepsilon^2}\sum\limits_{t=T_1,r_t>t^{-\kappa}}^{T}\ln{\left(\frac{\det{A_{t-1}}}{\det{A_t}}\right)}\nonumber\\
&&\leq\frac{16}{\varepsilon^2}\left\{d\ln{T}- \ln\left(\det{A_{T_1}}\right)\right\}
\label{sum_bound_rt_large2}
\end{eqnarray}

\end{frame}

\begin{frame}{Regret Calculation}
To sum over the $r_t\leq t^{-\kappa}$ bounds we use the following result: 
\begin{equation}
\int\exp\{az^r\}\,dz=-\frac{z(-az^r)^{-\frac{1}{r}}}{r}\Gamma\{\frac{1}{r},-az^r\}
\label{incomplete_gamma}
\end{equation}
This yields:
\begin{eqnarray}
&&\sum\limits_{t=T_1,r_t\leq t^{-\kappa}}^{T}\exp\left\{1-\frac{\varepsilon^2}{36r_t\left(\sqrt{2\ln{\frac{2}{\delta_t}}}+1\right)^2}\right\}=\nonumber
\end{eqnarray}
\end{frame}

\begin{frame}{Regret Calculation}
\begin{eqnarray}
&&=\sum\limits_{t=T_1,r_t\leq t^{-\kappa}}^{T}\exp\left\{1-\frac{\varepsilon^2t^{\kappa}}{36\left(\sqrt{2\ln{\frac{2}{\delta_t}}}+1\right)^2}\right\}\leq\nonumber\\
&&\leq e\int_{T_1}^{T}\exp\left\{-\frac{\varepsilon^2t^{\kappa}}{36\left(\sqrt{2\ln{\frac{2}{\delta_t}}}+1\right)^2}\right\}\,dt=\nonumber
\end{eqnarray}
\end{frame}

\begin{frame}{Regret Calculation}
\begin{eqnarray}
&&=\frac{e36^{\frac{1}{\kappa}}\left(\sqrt{2\ln{\frac{2}{\delta_t}}}+1\right)^{\frac{2}{\kappa}}}{\kappa\varepsilon^{\frac{2}{\kappa}}}\nonumber\\
&&\left[\Gamma\left\{\frac{1}{\kappa},\frac{\varepsilon^2T_{1}^{\kappa}}{36\left(\sqrt{2\ln{\frac{2}{\delta_t}}}+1\right)^2}\right\}-\Gamma\left\{\frac{1}{\kappa},\frac{\varepsilon^2T^{\kappa}}{36\left(\sqrt{2\ln{\frac{2}{\delta_t}}}+1\right)^2}\right\}\right]\nonumber
\end{eqnarray}
\newline\newline
The development of the sum $\sum\limits_{t=T_1}^{T}\exp\left\{1-\frac{\varepsilon}{6r_t}\right\}$ is identical up to constants.
\end{frame}

\begin{frame}{Regret Calculation}
We are left with summing over the $r_{L_t}$ terms.


\end{frame}
\section{Regression Analysis}

\section{Summary}








































\end{document}