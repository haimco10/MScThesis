\section{Second Order Aggressive SHAMPO}


\begin{algorithm}[!h]
\begin{algorithmic}
   \State \textbf{Parameters:}  $b\in\mathbb{R}>0$.
   \State \textbf{Initialize:} $\vwi{i,0}=\vzero$, $A_0=I$ \\
   \For {$t=1,2, \ldots, n$} 
     \begin{enumerate}
     \nolineskips
     \item Observe $K$ instance vectors, $\vxiit$, ($i=1 \comdots K$).
     \item Compute  $\hat{p}_{i,t}=\vxiit^T \paren{A_{i,t-1}+\vxiit\vxiit^T}^{-1}\vw_{i,t-1}$.
     \item Predict $K$ labels, $\hyi{i,t}=\sign(\hat{p}_{i,t})$.
      \item Compute $r_{i,t} = \vxiit^T A_{i,t-1}^{-1}\vxiit$.
     \item Draw problem $J_t$  with the distribution:
     
      \begin{align}
     \pr{J_t=j} &= \frac{1}{D_{t}}\frac{1}{b+\paren{\Theta\paren{\abs{\hat{p}_{j,t}},r_{j,t}}}_+}, \nonumber\\
     D_t &= \sum_i \paren{b+\paren{\Theta\paren{\abs{\hat{p}_{i,t}},r_{i,t}}}_+}^{-1}. \nonumber
     \end{align}
     Where $m,i\in\{1,\ldots,K\}$ and $D_t$ is the normalization factor. \\
     Set $Z_{J,t}=1,~Z_{i,t}=0 , ~\forall i\ne J $.

     
     \If {$\Theta\paren{\abs{\hat{p}_{J_t,t}},r_{i,t}}\ge0$}
        \If {$y_{J_t,t}\ne{\hat{y}_{J_t,t}$}}
            \State $U_{J_t,t}=1$
         \Else
            \State $U_{J_t,t}=0$
        \EndIf
    \Else 
        \State set $U_{J_t,t}=1$
    \EndIf
     
     \item Update:
     \begin{align}
     &\vwi{J_t,t} = \vwi{J_t,t-1}+U_{J_t,t}\,  \vxi{J_t,t}  \yi{J_t,t}\,\label{perc_update}\\
     &A_{J_t,t}=A_{J_t,t-1}+ U_{J_t,t}\vxi{J_t,t}\vxi{J_t,t}^T\nonumber
     \end{align}

     
     \end{enumerate}
   \EndFor  
   \State {\bf Output}: $\vwi{i,n}$ for $i=1 \comdots K$.
\end{algorithmic}
\caption{Second order aggressive SHAMPO.}\label{alg:SHAMPO}
\end{algorithm}

All the algorithms that were shown so far, estimated the uncertainty of the algorithm prediction using 
the low margin rule. Another approach of estimating the certainty of prediction could be 
comparing the input example with all the previous labeled examples. If the 
algorithm already received similar examples before, and requested their labels, 
probably the prediction that is based on those similar examples, can be considered with high confidence.
\cite{crammer2014doubly} used the same approach in the selective 
sampling setting, by introducing a distribution that depends on both margin and this uncertainty measure.
An aggressive update was applied as well, based on the same measure.

We adapt this approach to our setting of multitask learning sharing a single annotator. First, we introduce 
one more quantity that is the projection of the vector $\vxiit$ on the inverse of the covariance-like matrix $A_{i,t-1}$ 
\[
r_{i,t} = \vxiit^{T}A_{i,t-1}^{-1}\vxiit
\]
We can think about this quantity as an uncertainty measure. Where large values of $r_{i,t}$ 
represent high uncertainty, whereas small value of the same quantity are related 
to low uncertainty, since the algorithm already received labeled examples that are close 
to $\vxiit$. In the edge case, when the example $\vxiit$ is orthogonal to all 
previous queried examples that came from the same task, $r_{i,t}=1$. 

\noindent
We define here the function
\begin{equation}
\Theta\paren{\abs{\hat{p}_{i,t}},r_{i,t}}=\paren{1+r_{i,t}}\hat{p}_{i,t}^2+2\abs{\hat{p}_{i,t}}-\frac{r_{i,t}}{1+r_{i,t}}.
\end{equation}
which depends on both certainty measurements: $\abs{\hat{p}_{i,t}}$ and 
$r_{i,t}$, and introduce the new probability over the tasks

\begin{equation}
  \pr{J_t=j} = \frac{1}{D_{t}}\frac{1}{b+\paren{\Theta\paren{\abs{\hat{p}_{j,t}},r_{j,t}}}_+}
\end{equation}
where $D_t$  is the normalization factor
\begin{equation*}
  D_t &=\sum_i\paren{b+\paren{\Theta\paren{\abs{\hat{p}_{i,t}},r_{i,t}}}_+}^{-1}.
\end{equation*}

\noindent
One can see that even though $\Theta\paren{\abs{\hat{p}_{i,t}},r_{i,t}}}$ function can be negative, we 
consider only its positive values, so we always get a positive values of $\pr{J_t=j}$ which make it a distribution.
At each round we draw a task $J_t$ using this distribution.
After we draw a task to query on, we decide if a model update should be done based on two quantities: 
the function $\Theta\paren{\abs{\hat{p}_{J_t,t}},r_{i,t}}}$ and the mistake indicator $M_{J_t,t}$.
If $\Theta\paren{\abs{\hat{p}_{i,t}},r_{J_t,t}}} < 0$, than an aggressive update 
is been made, independent on the prediction value ,else, we update only when there is a prediction mistake 
(i.e. $M_{J_t,t}=1$). 

\noindent
To understand this update rule, we examine the behavior of $\Theta\paren{\abs{\hat{p}_{i,t}},r_{i,t}}}$ 
function. This function, is quadratic in $\abs{\hat{p}_{J_t,t}}$, solving the 
inequality $\Theta\paren{\abs{\hat{p}_{i,t}},r_{J_t,t}}} < 0$ for $\abs{\hat{p}_{J_t,t}}$ 
leads to the folowing bound on the margin,
\begin{equation*}
  \abs{y_{i,t}\hat{p}_{i,t}} \le \frac{-1+\sqrt{1+r_{i,t}}}{1+r_{i,t}}.
\end{equation*}

This dynamic aggressive update threshold is actually upper bounded by a constant, since when $r_{i,t}$ is maximal, 
($r_{i,t}=1$), this aggressive threshold gets its highest value, 
$\frac{\sqrt{2}-1}{2}\approx 0.2$.
We don't have a full analysis on the second order aggressive algorithm in the 
sense of the mistake bounds yet, however, it was well analyzed in the selective 
sampling setting and shown remarkable empirical improvement over the rest of the algorithms in our 
experiments.


%\begin{proof} 
%\\ 
%Before proofing the Thm. we will use the next inequality. define $x\in[0,1]$
%\begin{equation}
%\sqrt{1-x}+\sqrt{1+x}\le2.
%\label{technical_inequality}
%\end{equation}
%From the concavity of $\sqrt{1+x}$ we see that $\sqrt{1+x}\le1+\half x$. 
%Using this inequality twice, we get $\sqrt{1-x}+\sqrt{1+x}\le1-\half x+1+\half x=2$.
%Define the regularized cumulative square loss of the updated rounds on the task 
%$i$ up to the round $t$
%\begin{equation*}
%\Phi_{i,t}(\vu_i)=\half\normt{\vu_i}+\half\sum_{s=1}^{t}{Z_{i,t}U_{i,t}\paren{y_{i,t}-\vu_i^T\vx_{i,t}}}^2.
%\end{equation*}
%Simlar to the proof of Thm .3 of Cesa-Bianchi et al ~\cite{cesa2006worst} and Forster ????,
%\begin{equation*}
%\begin{split}
%\half Z_{i,t}U_{i,t}\paren{y_{i,t}-\hat{p}_{i,t}}^{2}= &\inf_{\vu_i}{\Phi_{i,t+1}}(\vu_i)-\inf_{\vu_i}{\Phi_{i,t}}(\vu_i)+\frac{Z_{i,t}U_{i,t}}{2}\vxiit^TA_{i,t}^{-1}\vxiit\\
%&-\frac{Z_{i,t}U_{i,t}}{2}\vxiit^TA_{i,t-1}^{-1}\vxiit\hat{p}_{i,t}^2\\
%=&\inf_{\vu_i}{\Phi_{i,t+1}}(\vu_i)-\inf_{\vu_i}{\Phi_{i,t}}(\vu_i)+\frac{Z_{i,t}U_{i,t}}{2}\frac{r_{i,t}}{1+r_{i,t}}-\frac{Z_{i,t}U_{i,t}}{2}r_{i,t}\hat{p}_{i,t}^2
%\end{split}
%\end{equation*} 
%Now, we sum up the equation over $t$,
% 
%\begin{equation*}
%\begin{split}
%\half \sum_{t=1}^{n}Z_{i,t}U_{i,t}\paren{y_{i,t}-\hat{p}_{i,t}}^{2}=& \inf_{\vu_i}{\Phi_{i,n+1}}(\vu_i)+ \sum_{t=1}^{n}\frac{Z_{i,t}U_{i,t}}{2}\frac{r_{i,t}}{1+r_{i,t}}-\sum_{t=1}^{n}\frac{Z_{i,t}U_{i,t}}{2}r_{i,t}\hat{p}_{i,t}^2\\
%\le&\half\normt{\vu_i}+\half\sum_{t=1}^{n}{Z_{i,t}U_{i,t}\paren{y_{i,t}-\vu_i^T\vx_{i,t}}}^2+ \sum_{t=1}^{n}\frac{Z_{i,t}U_{i,t}}{2}\frac{r_{i,t}}{1+r_{i,t}}\\&-\sum_{t=1}^{n}\frac{Z_{i,t}U_{i,t}}{2}r_{i,t}\hat{p}_{i,t}^2.
%\end{split}
%\end{equation*} 
%For simplification we will define
%\begin{equation*}
%A_{i,n}=I+\sum_{t=1}^{n}{Z_{i,t}U_{i,t}}\vxiit\vxiit^T.
%\end{equation*}
%Expanding the squares we get,
%\begin{equation}
%\begin{split}
%\half \sum_{t=1}^{n}Z_{i,t}U_{i,t}&\paren{\hat{p}_{i,t}^2-2y_{i,t}\hat{p}_{i,t}-\frac{r_{i,t}}{1+r_{i,t}}+r_{i,t}\hat{p}_{i,t}^2}\\
%\le& \half\normt{\vu_i}+\frac{1}{2}\sum_{t=1}^{n}{Z_{i,t}U_{i,t}\vu_i^T\vx_{i,t}\vx_{i,t}^T\vu_i}-\sum_{t=1}^{n}{Z_{i,t}U_{i,t}\vu_i^T\vx_{i,t}y_{i,t}}\\
%&=\half \vu_i^T\paren{I+\sum_{t=1}^{n}{Z_{i,t}U_{i,t}}\vxiit\vxiit^T}\vu_i-\sum_{t=1}^{n}{Z_{i,t}U_{i,t}\vu_i^T\vx_{i,t}y_{i,t}}\\
%&=\half \vu_i^T A_{i,n} \vu_i-\sum_{t=1}^{n}{Z_{i,t}U_{i,t}\vu_i^T\vx_{i,t}y_{i,t}}.
%\end{split}
%\label{sec_order_1}
%\end{equation} 
%The vectors $\vu_i$ can be replaced with their scaled version, $c\vu_i$. 
%Introducing the trivial inequality, $1-x\le \max\braces{1-x,0}$ we get 
%\begin{equation}
%\begin{split}
%cZ_{i,t}U_{i,t}\paren{1-\vu_i^T\vx_{i,t}y_{i,t}} &\le cZ_{i,t}U_{i,t}\max\braces{1-\vu_i^T\vx_{i,t}y_{i,t},0}\\
%-cZ_{i,t}U_{i,t}\vu_i^T\vx_{i,t}y_{i,t}&\le-cZ_{i,t}U_{i,t}+c Z_{i,t}U_{i,t}\lossp{}\paren{\vu_i^T\vx_{i,t}y_{i,t}}.
%\label{sec_order_2}
%\end{split}
%\end{equation}
%Rearranging an plugging \eqref{sec_order_2} and \eqref{sec_order_1} 
%
%\begin{equation}
%\begin{split}
%\half \sum_{t=1}^{n}Z_{i,t}U_{i,t}&\paren{\hat{p}_{i,t}^2-2y_{i,t}\hat{p}_{i,t}-\frac{r_{i,t}}{1+r_{i,t}}+r_{i,t}\hat{p}_{i,t}^{2}+2c}\\
%&\le\frac{c^2}{2} \vu_i^T A_{i,n} \vu_i+c\sum_{t=1}^{n}{Z_{i,t}U_{i,t}\lossp{}\paren{\vu_i^T\vx_{i,t}y_{i,t}}}
%\label{sec_order_2}
%\end{split}
%\end{equation}
%Recall that $U_{i,t}=M_{i,t}+G_{i,t}$ we will split the inequality into two different cases. First we will take into consideration the cases when an error update was performed, i.e.$M_{i,t}=1$, in which we have $-y_{i,t}\hat{p}_{i,t}=\abs{\hat{p}_{i,t}}$. In this case we need to consider also two subcases, when $\Theta\paren{\abs{\hat{p}_{i,t}},r_{i,t}}\ge0$   and when $\Theta\paren{\abs{\hat{p}_{i,t}},r_{i,t}}<0$. Beginning with the former subcase, recall that for this case 
%$\mathbb{E}_{t-1}\brackets{Z_{i,t}}=\frac{1}{D_{t}}\frac{2c}{2c+\paren{\Theta\paren{\abs{\hat{p}_{j,t}},r_{j,t}}}}$, we get 
%\begin{equation*}
%\begin{split}
%\mathbb{E}&\brackets{Z_{i,t}U_{i,t}\paren{\hat{p}_{i,t}^2-2y_{i,t}\hat{p}_{i,t}-\frac{r_{i,t}}{1+r_{i,t}}+r_{i,t}\hat{p}_{i,t}^{2}+2c}}\\
%&=\mathbb{E}\brackets{\mathbb{E}_{t-1}\brackets{Z_{i,t}}U_{i,t}\paren{\hat{p}_{i,t}^2-2y_{i,t}\hat{p}_{i,t}-\frac{r_{i,t}}{1+r_{i,t}}+r_{i,t}\hat{p}_{i,t^{}}^{2}+2c}}\\
%&=2c\mathbb{E}\brackets{\frac{1}{D_t}U_{i,t}}.
%\end{split}
%\end{equation*}
%When $\Theta\paren{\abs{\hat{p}_{i,t}},r_{i,t}}<0$  the conditional expectation becomes $\mathbb{E}_{t-1}\brackets{Z_{i,t}}=\frac{1}{D_t}$ and the thus,
%\begin{equation*}
%\begin{split}
%\mathbb{E}&\brackets{Z_{i,t}U_{i,t}\paren{\hat{p}_{i,t}^2-2y_{i,t}\hat{p}_{i,t}-\frac{r_{i,t}}{1+r_{i,t}}+r_{i,t}\hat{p}_{i,t}^{2}+2c}}\\
%&=\mathbb{E}\brackets{\mathbb{E}_{t-1}\brackets{Z_{i,t}}U_{i,t}\paren{\hat{p}_{i,t}^2+2\abs{\hat{p}_{i,t}}-\frac{r_{i,t}}{1+r_{i,t}}+r_{i,t}\hat{p}_{i,t}^{2}+2c}}\\
%&\ge\mathbb{E}\brackets{\mathbb{E}_{t-1}\brackets{Z_{i,t}}U_{i,t}\paren{-\frac{r_{i,t}}{1+r_{i,t}}+2c}}\\
%&\ge 2c 
%\mathbb{E}\brackets{\frac{U_{i,t}}{D_t}}-\frac{r_{i,t}}{1+r_{i,t}}\mathbb{E}\brackets{\frac{U_{i,t}}{D_t}}\\
%&\ge 2c \mathbb{E}\brackets{\frac{1}{D_t}U_{i,t}}-\frac{r_{i,t}}{1+r_{i,t}}\mathbb{E}\brackets{\frac{1}{D_t}}.
%\end{split}
%\end{equation*}
%Now we examine the case where an update was performed, but there was no mistake. In this case, $0\le y_{i,t}\hat{p}_{i,t}$ and the aggressive update was performed. Recall the bound on the margin for such case and using \eqref{sec_order_2} ,
%we bound the margin as follows\begin{equation*}
%0\le y_{i,t}\hat{p}_{i,t}\le \theta({r_{i,t}})=\frac{-1+\sqrt{1+r_{i,t}}}{1+r_{i,t}}.
%\end{equation*}
%We can bound now,
%\begin{equation*}
%\begin{split}
%\hat{p}_{i,t}^2&-2y_{i,t}\hat{p}_{i,t}-\frac{r_{i,t}}{1+r_{i,t}}+r_{i,t}\hat{p}_{i,t}^{2}+2c\\
%&=(1+r_{i,t})\hat{p}_{i,t}^2-2y_{i,t}\hat{p}_{i,t}+\frac{r_{i,t}}{1+r_{i,t}}-2\frac{r_{i,t}}{1+r_{i,t}}+2c\\
%&=f(y_{i,t}\hat{p}_{i,t})-2\frac{r_{i,t}}{1+r_{i,t}}+2c
%\end{split}
%\end{equation*}
%\end{proof}          
%where $f(y_{i,t}\hat{p}_{i,t})=(1+r_{i,t})\hat{p}_{i,t}^2-2y_{i,t}\hat{p}_{i,t}+\frac{r_{i,t}}{1+r_{i,t}}$ is a quadratic convex function with two non-negative roots $\frac{1\pm\sqrt{1-r_{i,t}}}{1+r_{i,t}}$ and we know that the margin is lower than the smaller root , $y_{i,t}\hat{p}_{i,t}\le\frac{1-\sqrt{1-r_{i,t}}}{1+r_{i,t}}$ which leads to the inequality $f(y_{i,t}\hat{p}_{i,t})\ge0$ so we bound 
%\begin{equation*}
%\begin{split}
%\mathbb{E}&\brackets{Z_{i,t}U_{i,t}\paren{\hat{p}_{i,t}^2-2y_{i,t}\hat{p}_{i,t}-\frac{r_{i,t}}{1+r_{i,t}}+r_{i,t}\hat{p}_{i,t}^{2}+2c}}\\
%&=\mathbb{E}\brackets{\mathbb{E}_{t-1}\brackets{Z_{i,t}}U_{i,t}\paren{f(y_{i,t}\hat{p}_{i,t})-2\frac{r_{i,t}}{1+r_{i,t}}+2c}}\\
%&\ge\mathbb{E}\brackets{\mathbb{E}_{t-1}\brackets{Z_{i,t}}U_{i,t}\paren{-2\frac{r_{i,t}}{1+r_{i,t}}+2c}}\\
%&\ge 2c\mathbb{E}\brackets{\frac{U_{i,t}}{D_t}}-\frac{2r_{i,t}}{1+r_{i,t}}\mathbb{E}\brackets{\frac{U_{i,t}}{D_t}}
%&\ge 2c\mathbb{E}\brackets{\frac{1}{D_t}U_{i,t}}-\frac{2r_{i,t}}{1+r_{i,t}}\mathbb{E}\brackets{\frac{1}{D_t}}.
%\end{split}
%\end{equation*}
%Summarize the results we get,
%\begin{equation}
%\begin{split}
%\half \sum_{t=1}^{n}&\mathbb{E}\brackets{Z_{i,t}U_{i,t}\paren{\hat{p}_{i,t}^2-2y_{i,t}\hat{p}_{i,t}-\frac{r_{i,t}}{1+r_{i,t}}+r_{i,t}\hat{p}_{i,t}^{2}+2c}}\\
%&\ge c \sum_{t\in\mathcal{M}}\mathbb{E}\brackets{\frac{1}{D_t}U_{i,t}}+c \sum_{t\in\mathcal{G}}\mathbb{E}\brackets{\frac{1}{D_t}U_{i,t}}\\
%&-\half\sum_{t\in\mathcal{A\cap M}}\frac{r_{i,t}}{1+r_{i,t}}\mathbb{E}\brackets{\frac{1}{D_t}}
%-\sum_{t\in\mathcal{A\cap G}}\frac{r_{i,t}}{1+r_{i,t}}\mathbb{E}\brackets{\frac{1}{D_t}}
%\end{split}
%\end{equation}
%Combining the result of the last inequality with the expectation of the  \eqref{sec_order_2}, recall that  $\sum_{t\in\mathcal{M}}U_{i,t}=M_i$ , and  $\sum_{t\in\mathcal{G}}U_{i,t}=G_i$ we get,   
%\begin{equation*}
%\begin{split}
%c \sum_{t\in\mathcal{M}}\mathbb{E}\brackets{\frac{1}{D_t}U_{i,t}}+c \sum_{t\in\mathcal{G}}\mathbb{E}&\brackets{\frac{1}{D_t}U_{i,t}}
%-\frac{1}{2}\sum_{t\in\mathcal{A\cap M}}\frac{r_{i,t}}{1+r_{i,t}}\mathbb{E}\brackets{\frac{1}{D_t}}\\
%-\sum_{t\in\mathcal{A\cap U}}\frac{r_{i,t}}{1+r_{i,t}}\mathbb{E}\brackets{\frac{1}{D_t}}
%\le& \frac{c^2}{2} \vu_i^T \mathbb{E}\brackets{A_{i,n}} \vu_i+c\sum_{t=1}^{n}\mathbb{E}\brackets{{Z_{i,t}U_{i,t}\lossp{}\paren{\vu_i^T\vx_{i,t}y_{i,t}}}}
%\end{split}
%\end{equation*}
%
%The normalization can also been bound by
%\begin{equation*}
%\begin{split}
%D_{t}=2c\sum_{i=1}^{K}{\left({2c+\paren{\Theta\paren{\abs{\hat{p}_{i,t}},r_{i,t}}}_+}\right)^{-1}}& \le  2c\sum_{m=1}^{K}{\frac{1}{2c}}=K
%\end{split}
%\end{equation*}
%which leads to
%
%
%\begin{equation*}
%\begin{split}
%\mathbb{E}\brackets{M_{i}}+ &\mathbb{E}\brackets{G_{i}}
%-\frac{1}{2c}\mathbb{E}\sum_{t\in\mathcal{A\cap M}}\brackets{\frac{r_{i,t}}{1+r_{i,t}}}
%-\frac{1}{c}\mathbb{E}\sum_{t\in\mathcal{A\cap U}}\brackets{\frac{r_{i,t}}{1+r_{i,t}}}\\
%\le& \frac{Kc}{2} \vu_i^T \mathbb{E}\brackets{A_{i,n}} \vu_i+K\sum_{t=1}^{n}\mathbb{E}\brackets{{Z_{i,t}U_{i,t}\lossp{}\paren{\vu_i^T\vx_{i,t}y_{i,t}}}}
%\end{split}
%\end{equation*}
%
%
%
%\begin{equation*}
%\begin{split}
%\mathbb{E}\brackets{M_{i}}\le&\frac{Kc}{2} \vu_i^T \mathbb{E}\brackets{A_{i,n}} \vu_i+K\sum_{t=1}^{n}\mathbb{E}\brackets{{Z_{i,t}U_{i,t}\lossp{}\paren{\vu_i^T\vx_{i,t}y_{i,t}}}} -\mathbb{E}\brackets{G_{i}}\\
%&+\frac{1}{2c}\mathbb{E}\sum_{t\in\mathcal{A\cap M}}\brackets{\frac{r_{i,t}}{1+r_{i,t}}}
%+\frac{1}{c}\mathbb{E}\sum_{t\in\mathcal{A\cap U}}\brackets{\frac{r_{i,t}}{1+r_{i,t}}}\\
%\le& \frac{Kc}{2} \vu_i^T \mathbb{E}\brackets{A_{i,n}} \vu_i+K\mathbb{E}\brackets{\sum_{t=1}^{n}{Z_{i,t}U_{i,t}\lossp{}\paren{\vu_i^T\vx_{i,t}y_{i,t}}}}-\mathbb{E}\brackets{G_{i}}\\
%&+\frac{1}{c}\sum_{t\in\mathcal{A}}\brackets{\frac{r_{i,t}}{1+r_{i,t}}}
%\end{split}
%\end{equation*}
%Now, we can summarize the inequality over all of the tasks,
%\begin{equation*}
%\begin{split}
%\mathbb{E}\brackets{M}\le &\frac{cK}{2}\sum_{i=1}^{K}  \vu_i^T \mathbb{E}\brackets{A_{i,n}} \vu_i+K\mathbb{E}\brackets{\sum_{i=1}^{K}\sum_{t=1}^{n}Z_{i,t}U_{i,t}{\lossp{}\paren{\vu_i^T\vx_{i,t}y_{i,t}}}}-\mathbb{E}\brackets{G}\\
%&+\frac{1}{c}\mathbb{E}\sum_{i=1}^{K}\sum_{t\in\mathcal{A}}\brackets{\frac{r_{i,t}}{1+r_{i,t}}}
%\end{split}
%\end{equation*}
% 
%\begin{equation*}
%\begin{split}
%\mathbb{E}\brackets{M}\le &\frac{cK}{2}\sum_{i=1}^{K}  \vu_i^T \mathbb{E}\brackets{A_{i,n}} \vu_i+K{\bar L}_{1,n}-\mathbb{E}\brackets{G}\\
%&+\frac{1}{c}\mathbb{E}\sum_{i=1}^{K}\sum_{t\in\mathcal{A}}\brackets{\frac{r_{i,t}}{1+r_{i,t}}}
%\end{split}
%\end{equation*} 
%


