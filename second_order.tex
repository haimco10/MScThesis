\chapter{Second Order Shampo}

\fbox{replace matrix $A$ with another letter to avoid confusion}
The second order version of the perceptron algorithm, was proposed and analyzed by 
\cite{CesaBianchiCoGe05}. This algorithm was adopted to the online binary classification 
settings from the ridge-regression (squared loss and Euclidean regularization) framework as described by 
\cite{hoerl1970ridge} and \cite{Vovk97}, and further analyzed in \cite{AzouryWa01} and  \cite{forster2002relative}. 
It was also shown that this variation has the effect of reducing the number of mistakes. 
\cite{DBLP:journals/jmlr/Cesa-BianchiGZ06a} also adapted the second order 
perceptron algorithm to the selective sampling case which was extended later by \cite{crammer2014doubly}.

In this algorithm, we store and update two quantities per task instead of one, the matrix 
$A_{i,t} = I+\sum_{t}M_{i,t}Z_{i,t}\vxiit\vxiit^T$ and the vector $\vwi{i,t} = \sum_{t}M_{i,t}Z_{i,t} \vxi{i,t}  \yi{i,t}$.
Whereas, the probability to issue a query on task $i$ in time $t$ is similar to the first order SHAMPO algorithm,
the update  that is done here in the case of queried mistake, is the second order perceptron update.  
The pseudo code of the second order SHAMPO is shown in \algoref{alg:SO_SHAMPO}


\begin{algorithm}
\begin{algorithmic}
   \State \textbf{Parameters:}  $b\in\mathbb{R}>0$.
   \State \textbf{Initialize:} $\vwi{i,0}=\vzero$, $A_{i,0}=I$ \\
   \For {$t=1,2, ..., n$} 
     \begin{enumerate}
     \nolineskips
     \item Observe $K$ instance vectors, $\vxiit$, ($i=1 \comdots K$).
     \item Compute  $\hat{p}_{i,t}=\vxiit^T \paren{A_{i,t-1}+\vxiit\vxiit^T}^{-1}\vw_{i,t-1}$.
     \item Predict $K$ labels, $\hyi{i,t}=\sign(\hat{p}_{i,t})$.
     \item Draw problem $J_t$  with the distribution:
      \begin{align}
    \pr{J_t=j} &=
    \frac{\paren{b+\abs{\hat{p}_{j,t}}-\min_{m=1}^K\abs{\hat{p}_{m,t}}}^{-1}}{D_{t}},\nonumber\\
    D_t &=\sum_i \paren{b+\abs{\hat{p}_{i,t}}-\min_{m=1}^K\abs{\hat{p}_{m,t}}}^{-1}. \nonumber
     \end{align}
     Where $m,i\in\{1,...,K\}$ and $D_t$ is the normalization factor. 
  \item Query the true label ,$\yi{J_t,t}\in\{-1,1\}$.
  \item Set the indicator $M_{J_t, t}=1$ iff $\yi{J_t,t} \neq \hyi{J_t,t}$.
  \item Update with the perceptron rule:     
     \item Update:
     \begin{align}
     &\vwi{J_t,t} = \vwi{J_t,t-1}+M_{J_t,t}\,  \vxi{J_t,t}  \yi{J_t,t}\,\label{SO_update}\\
     &A_{J_t,t}=A_{J_t,t-1}+ M_{J_t,t}\vxi{J_t,t}\vxi{J_t,t}^T\nonumber
     \end{align}
     \end{enumerate}
   \EndFor  
   \State {\bf Output}: $\vwi{i,n}$ for $i=1 \comdots K$.
\end{algorithmic}
\caption{Second order aggressive SHAMPO }\label{alg:SO_SHAMPO}
\end{algorithm}



\begin{theorem}
  If second order SHAMPO algorithm runs on $K$ tasks with $K$ parallel example pair
  sequences
  $(\vxi{i,1},y_{i,1}),...(\vxi{i,n},y_{i,n})\in\mathbb{R}^d\times\{-1,1\}$,
  $i=1,...,K$ with input parameter $b>0$, then for all $\gamma>0$, all
  $\vui{i}\in\mathbb{R}^d$ and all $n\ge1$, there exists $0<\delta\le K$, such that,
  \begin{equation*}
  \begin{split}
   &\mathbb{E}\brackets{\sum_{i=1}^{K}\sum_{t=1}^{n}{M_{i,t}}} \\
   &\le \frac{\delta}{\gamma}{\bar L}_{\gamma,n}(\vui{i})
+ \frac{\delta b}{2\gamma^2}\sum_{i=1}^{K}\vu_i^T\mathbb{E}\brackets{A_{i,n}}\vu_i+ 
\frac{\delta}{2b}\sum_{i=1}^{K}\sum_{k=1}^{d}\mathbb{E}\brackets{\ln\paren{1+\lambda_{i,k}}},
\end{split}
\end{equation*} 
 where $A_{i,n} = I+\sum_{t=1}^{n}M_{i,t}Z_{i,t}\vxiit\vxiit^T$ , $\lambda_{i,k}$ is the $i^th$ eigenvalue
 of the matrix $A_{i,n}$ and the expectation is over the random choices of the algorithm.
\end{theorem} \label{thm:SO_bound}

\begin{proof}
Define the regularized cumulative square loss of the updated rounds on the task 
$i$ up to the round $t$
\begin{equation*}
\Phi_{i,t}(\vu_i)=\half\normt{\vu_i}+\half\sum_{s=1}^{t}{Z_{i,t}M_{i,t}\paren{y_{i,t}-\vu_i^T\vx_{i,t}}}^2.
\end{equation*}
It was proved by \cite{Forster} in the linear regression case and adopted by
~\cite{cesa2006worst} and Crammer \cite{crammer2014doubly} for the second order classification that,
\begin{equation*}
\begin{split}
\half Z_{i,t}M_{i,t}\paren{y_{i,t}-\hat{p}_{i,t}}^{2}= &\inf_{\vu_i}{\Phi_{i,t+1}}(\vu_i)-\inf_{\vu_i}{\Phi_{i,t}}(\vu_i)
+\frac{Z_{i,t}M_{i,t}}{2}\vxiit^TA_{i,t}^{-1}\vxiit\\
&-\frac{Z_{i,t}M_{i,t}}{2}\vxiit^TA_{i,t-1}^{-1}\vxiit\hat{p}_{i,t}^2.
\end{split}
\end{equation*} 
We can drop now the last term which is nonnegative because $A_{i,t-1}$ is 
positive definite matrix and so that $A_{i,t-1}^{-1}$.
Now, we sum up the equation over $t$ and get

\begin{equation*}
 \begin{split}
\sum_{t=1}^{n}\frac{Z_{i,t}M_{i,t}}{2}&\paren{y_{i,t}-\hat{p}_{i,t}}^{2} \le \inf_{\vu_i}{\Phi_{i,n+1}}(\vu_i)
-\inf_{\vu_i}{\Phi_{i,1}}(\vu_i)+\sum_{t=1}^{n}\frac{Z_{i,t}M_{i,t}}{2}\vxiit^TA_{i,t}^{-1}\vxiit\\
&\le  \Phi_{i,n+1}(\vu_i)+\sum_{t=1}^{n}\frac{Z_{i,t}M_{i,t}}{2}\vxiit^TA_{i,t}^{-1}\vxiit\\
&\le  \half\normt{\vu_i}+\half\sum_{t=1}^{n}{Z_{i,t}M_{i,t}\paren{y_{i,t}-\vu_i^T\vx_{i,t}}}^2+
\sum_{t=1}^{n}\frac{Z_{i,t}M_{i,t}}{2}\vxiit^TA_{i,t}^{-1}\vxiit\\
\end{split}
\end{equation*} 
since $\inf_{\vu_i}{\Phi_{i,1}}(\vu_i)=0$. We now expand the squares and the 
inequality becomes

\begin{equation*}
 \begin{split}
   \sum_{t=1}^{n}&\frac{Z_{i,t}M_{i,t}}{2}\paren{\hat{p}_{i,t}^2-2\hat{p}_{i,t}y_{i,t}}
   \le  \half\normt{\vu_i} \\
   &+\half\sum_{t=1}^{n}{Z_{i,t}M_{i,t}\paren{\vu_i^T\vx_{i,t}}}^2 
   +\sum_{t=1}^{n}{Z_{i,t}M_{i,t}\vu_i^T\vx_{i,t}y_{i,t}}
+\sum_{t=1}^{n}\frac{Z_{i,t}M_{i,t}}{2}\vxiit^TA_{i,t}^{-1}\vxiit.\\
\end{split}
\end{equation*} 

Now we handle the right hand side of the inequality one by one. First we start by writing the two first terms 
in the form 
\begin{equation*}
\begin{split}
  \half\normt{\vu_i}+\half\sum_{t=1}^{n}{Z_{i,t}M_{i,t}\paren{\vu_i^T\vx_{i,t}}}^2  
  &= \half\vu_i^T\paren{I+\half\sum_{t=1}^{n}{Z_{i,t}M_{i,t}\vx_{i,t}\vx_{i,t}^T}}\vu_i\\
  &=\half\vu_i^TA_{i,n}\vu_i
\end{split}
\end{equation*}
% https://books.google.co.il/books?hl=en&lr=&id=Pd8bQNTjLEcC&oi=fnd&pg=PA269&ots=IU3lTn2wbb&sig=xFEXkSmP51J1rmXlFvCMa1SCXYU&redir_esc=y#v=onepage&q&f=false
Denote the $k_i^{th}$ eigenvalue of the matrix $A_{i,n}$ by $1+\lambda_{i,k}$, as proved by \cite{Forster},
we  bound the last term  by

\begin{equation*}
\begin{split}
  \half\sum_{t=1}^{n}Z_{i,t}M_{i,t}\vxiit^TA_{i,t}^{-1}\vxiit 
  &\le \half\sum_{t=1}^{n}\ln\paren{\frac{\det{A_{i,t}}}{\det{A_{i,t-1}}}} 
  &= \half\ln\paren{\frac{\det{A_{i,n}}}{\det{A_{i,0}}}}
  &=\half\ln\paren{\det{A_{i,n}}} = \half\sum_{k=1}^{d}\ln\paren{1+\lambda_{i,k}}.
\end{split}
\end{equation*}

Plugging the terms back into the inequality and dropping the positive term from 
the left hand side, recall that on round when there is a mistake ($M_{i,t}=1$), $\hat{p}_{i,t}y_{i,t}\le0$ 
, we obtain 
\begin{equation*}
   \sum_{t=1}^{n}&Z_{i,t}M_{i,t}\paren{\abs{\hat{p}_{i,t}}+\vu_i^T\vx_{i,t}y_{i,t}}
   \le \half\vu_i^TA_{i,n}\vu_i+ \half\sum_{k=1}^{d}\ln\paren{1+\lambda_{i,k}}.
\end{equation*} 
By the definition of hinge loss, $\gamma-\lossp{\gamma,i,t}(\vu_i) \le y_{i,t}\vu_i^T\vxiit 
$. Replacing the $vu_i$ with it's scaling $\frac{b}{\gamma}\vu_i$ yields 
\begin{equation*}
    \begin{split}
   \sum_{t=1}^{n}Z_{i,t}M_{i,t}&\paren{\abs{\hat{p}_{i,t}}+b}\\
   &\le \frac{b}{\gamma}\sum_{t=1}^{n}Z_{i,t}M_{i,t}\lossp{\gamma,i,t}(\vu_i) 
   + \frac{b^2}{2\gamma^2}\vu_i^TA_{i,n}\vu_i+ \half\sum_{k=1}^{d}\ln\paren{1+\lambda_{i,k}}.
  \end{split}
\end{equation*} 
Now, we subtract a non negative quantity $\sum_{t=1}^{n}M_{i,t}Z_{i,t}
\min_j{\abs{\hat{p}_{j,t}}}$ from the left hand side and get,
\begin{equation*}
  \begin{split}
   \sum_{t=1}^{n}Z_{i,t}M_{i,t}&\paren{\abs{\hat{p}_{i,t}}-\min_j{\abs{\hat{p}_{j,t}}}+b}\\
   &\le \frac{b}{\gamma}\sum_{t=1}^{n}Z_{i,t}M_{i,t}\lossp{\gamma,i,t}(\vu_i) 
   + \frac{b^2}{2\gamma^2}\vu_i^TA_{i,n}\vu_i+ \half\sum_{k=1}^{d}\ln\paren{1+\lambda_{i,k}}.
\end{split}
\end{equation*} 
Now we take the expectation on both inequality sides. First we start from the 
left side. As in the first order proof, 
\begin{equation*}
   \mathbb{E}\brackets{\sum_{t=1}^{n}&Z_{i,t}M_{i,t}\paren{\abs{\hat{p}_{i,t}}-\min_j{\abs{\hat{p}_{j,t}}}+b}}
=\mathbb{E}\brackets{\sum_{t=1}^{n}{\frac{M_{i,t}}{D_{t}}}}.
\end{equation*} 
Taking the expectation from the right hand side the equation becomes
\begin{equation*}
  \begin{split}
   \mathbb{E}\brackets{\sum_{t=1}^{n}{\frac{M_{i,t}}{D_{t}}}}
   \le \frac{b}{\gamma}{\bar L}_{\gamma,i,n}(\vui{i})
   + \frac{b^2}{2\gamma^2}\vu_i^T\mathbb{E}\brackets{A_{i,n}}\vu_i+ \half\sum_{k=1}^{d}\mathbb{E}\brackets{\ln\paren{1+\lambda_{i,k}}}.
\end{split}
\end{equation*} 
We use now \eqref{eq:introducing_delta} and 
\begin{equation*}
  \begin{split}
   \frac{b}{\delta_i}\mathbb{E}\brackets{\sum_{t=1}^{n}{M_{i,t}}}
   \le \frac{b}{\gamma}{\bar L}_{\gamma,i,n}(\vui{i})
+ \frac{b^2}{2\gamma^2}\vu_i^T\mathbb{E}\brackets{A_{i,n}}\vu_i+ \half\sum_{k=1}^{d}\mathbb{E}\brackets{\ln\paren{1+\lambda_{i,k}}}
\end{split}
\end{equation*} 

Summing up the last inequality over all K tasks and setting $\delta = \max{\delta_i}$ yields,

\begin{equation*}
  \begin{split}
   \frac{1}{\delta}&\mathbb{E}\brackets{\sum_{i=1}^{K}\sum_{t=1}^{n}{M_{i,t}}} \\
   &\le \frac{1}{\gamma}{\bar L}_{\gamma,n}(\vui{i})
+ \frac{b}{2\gamma^2}\sum_{i=1}^{K}\vu_i^T\mathbb{E}\brackets{A_{i,n}}\vu_i+ \frac{1}{2b}\sum_{i=1}^{K}\sum_{k=1}^{d}\mathbb{E}\brackets{\ln\paren{1+\lambda_{i,k}}}
\end{split}
\end{equation*} 
\end{proof}









