\chapter{Second Order SHAMPO}

% \section{second order aggressive}
% 
%   \begin{algorithmic}
%     \State \textbf{Parameters:}  $c\in\mathbb{R}>0$.\\ 
%     \State \textbf{Initialize:} $\vwi{i,0}=\vzero$, $A_0=I$ \\
%     \For {$t=1,2, ..., n$} 
%       
% \begin{enumerate}
% \nolineskips
% \item Observe $K$ instance vectors, $\vxiit$, ($i=1 \comdots K$).
% \item Compute  $\hat{p}_{i,t}=\vxiit^T \paren{A_{i,t-1}+\vxiit\vxiit^T}^{-1}\vw_{i,t-1}$.
% \item Predict $K$ labels, $\hyi{i,t}=\sign(\hat{p}_{i,t})$.
% \item Compute $r_i = \vxiit^T A_{i,t-1}^{-1}\vxiit$.
% \item Draw problem $J_t$  with the distribution:
% \begin{align}
% \pr{J_t=j} &=
% \frac{1}{D_{t}}\frac{2c}{2c+\paren{\Theta\paren{\abs{\hat{p}_{j,t}},r_{j,t}}}_+}, \nonumber\\
% D_t &=2c
% \sum_i \paren{2c+\paren{\Theta\paren{\abs{\hat{p}_{i,t}},r_{i,t}}}_+}^{-1}. \nonumber
% \end{align}
% %Where $m,i\in\{1,...,K\}$ and $D_t$ is the normalization factor. Set $Z_{J,t}=1,~Z_{i,t}=0 , ~\forall i\ne J $.
% If  \Theta\paren{\abs{\hat{p}_{J_t,t}},r_{i,t}}\ge0 \Then
%         
%         if $y_{J_t,t}\ne{\hat{y}_{J_t,t}$ set $U_{J_t,t}=1$.
%         
%         end if 
% 
% else
% 
% set $U_{J_t,t}=1$.
% 
% end if.
% 
% \item Update:
% \begin{align}
% &\vwi{J_t,t} = \vwi{J_t,t-1}+U_{J_t,t}\,  \vxi{J_t,t}  \yi{J_t,t}\,\label{perc_update}\\
% &A_{J_t,t}=A_{J_t,t-1}+ U_{J_t,t}\vxi{J_t,t}\vxi{J_t,t}^T\nonumber
% \end{align}
% \end{enumerate}
% \EndFor  
% \State {\bf Output}: $\vwi{i,n}$ for $i=1 \comdots K$.
% \end{algorithmic}\\

% \caption{Second order aggressive SHAMPO. \label{alg:SHAMPO}}
% \end{figure}        
%\end{algorithm}


This algorithm and proof builds on the second order selective sampling by Cesa-Bianchi et al ~\cite{cesa2006worst} define the function
\begin{equation}
\Theta\paren{\abs{\hat{p}_{i,t}},r_{i,t}}=\paren{1+r_{i,t}}\hat{p}_{i,t}^2+2\abs{\hat{p}_{i,t}}-\frac{r_{i,t}}{1+r_{i,t}}
\end{equation}

In order to understand the way that the aggressive update works, there is a need to examine the function $\Theta\paren{\abs{\hat{p}_{i,t}},r_{i,t}}$. One can see that this function is quadratic in $\hat{p}_{i,t}$ , such that it becomes negative in a close interval. Solving this quadratic form, shows that 
\begin{equation}
\Theta\paren{\abs{\hat{p}_{i,t}},r_{i,t}}\le0 \Leftrightarrow \abs{\hat{p}_{i,t}}\le\theta({r_{i,t}})=\frac{-1+\sqrt{1+r_{i,t}}}{1+r_{i,t}}.
\end{equation}
 The last equation shows that the condition on the function $\Theta\paren{\abs{\hat{p}_{j,t}},r_{j,t}}$ can be translated to threshold on the margin. First, a task is chosen according to the distribution above, then, for margin that is less the threshold, an aggressive update will be issued whereas for margin that is bigger than that, the algorithm will performed update only when there is a prediction error. For 
all the rest of the task that were not chosen, no update is issued. For now, we would like to analyse the possibilities for one task in a certain iteration. There are two extreme cases: First, when the the algorithm has not get yet input examples the similar to the current one, which cause the maximal uncertainty i.e., $r_{i,t}=1$. In this case,  the aggressiveness threshold is maximal as well, and we get $\max{\theta({r_{i,t}})}=\frac{-1+\sqrt{2}}{2}\approx0.21$ and an aggressive update will be issued only when the margin is less than this value. Another interesting case is when we saw the same example many times before, such the uncertainty in the prediction now is relatively low, which means $r_{i,t}\approx0$. In this scenario, the aggressive update will be issued only when the margin is zero, $\hat{p}_{i,t}=0$. 
\\

\begin{proof} 
\\ 
Before proofing the Thm. we will use the next inequality. define $x\in[0,1]$
\begin{equation}
\sqrt{1-x}+\sqrt{1+x}\le2.
\label{technical_inequality}
\end{equation}
From the concavity of $\sqrt{1+x}$ we see that $\sqrt{1+x}\le1+\half x$. using this inequality twice, we get $\sqrt{1-x}+\sqrt{1+x}\le1-\half x+1+\half x=2$.

Define 
\begin{equation*}
\Phi_{i,t}(\vu_i)=\half\normt{\vu_i}+\half\sum_{s=1}^{t}{Z_{i,t}U_{i,t}\paren{y_{i,t}-\vu_i^T\vx_{i,t}}}^2
\end{equation*}
Simlar to the proof of Thm .3 of Cesa-Bianchi et al ~\cite{cesa2006worst} and Forster ????,
\begin{equation*}
\begin{split}
\half Z_{i,t}U_{i,t}\paren{y_{i,t}-\hat{p}_{i,t}}^{2}= &\inf_{\vu_i}{\Phi_{i,t+1}}(\vu_i)-\inf_{\vu_i}{\Phi_{i,t}}(\vu_i)+\frac{Z_{i,t}U_{i,t}}{2}\vxiit^TA_{i,t}^{-1}\vxiit\\
&-\frac{Z_{i,t}U_{i,t}}{2}\vxiit^TA_{i,t-1}^{-1}\vxiit\hat{p}_{i,t}^2\\
=&\inf_{\vu_i}{\Phi_{i,t+1}}(\vu_i)-\inf_{\vu_i}{\Phi_{i,t}}(\vu_i)+\frac{Z_{i,t}U_{i,t}}{2}\frac{r_{i,t}}{1+r_{i,t}}-\frac{Z_{i,t}U_{i,t}}{2}r_{i,t}\hat{p}_{i,t}^2
\end{split}
\end{equation*} 
Now, we sum up the equation over $t$,
 
\begin{equation*}
\begin{split}
\half \sum_{t=1}^{n}Z_{i,t}U_{i,t}\paren{y_{i,t}-\hat{p}_{i,t}}^{2}=& \inf_{\vu_i}{\Phi_{i,n+1}}(\vu_i)+ \sum_{t=1}^{n}\frac{Z_{i,t}U_{i,t}}{2}\frac{r_{i,t}}{1+r_{i,t}}-\sum_{t=1}^{n}\frac{Z_{i,t}U_{i,t}}{2}r_{i,t}\hat{p}_{i,t}^2\\
\le&\half\normt{\vu_i}+\half\sum_{t=1}^{n}{Z_{i,t}U_{i,t}\paren{y_{i,t}-\vu_i^T\vx_{i,t}}}^2+ \sum_{t=1}^{n}\frac{Z_{i,t}U_{i,t}}{2}\frac{r_{i,t}}{1+r_{i,t}}\\&-\sum_{t=1}^{n}\frac{Z_{i,t}U_{i,t}}{2}r_{i,t}\hat{p}_{i,t}^2.
\end{split}
\end{equation*} 
For simplification we will define
\begin{equation*}
A_{i,n}=I+\sum_{t=1}^{n}{Z_{i,t}U_{i,t}}\vxiit\vxiit^T.
\end{equation*}
Expanding the squares we get,
\begin{equation}
\begin{split}
\half \sum_{t=1}^{n}Z_{i,t}U_{i,t}&\paren{\hat{p}_{i,t}^2-2y_{i,t}\hat{p}_{i,t}-\frac{r_{i,t}}{1+r_{i,t}}+r_{i,t}\hat{p}_{i,t}^2}\\
\le& \half\normt{\vu_i}+\sum_{t=1}^{n}{Z_{i,t}U_{i,t}\vu_i^T\vx_{i,t}\vx_{i,t}^T\vu_i}-\sum_{t=1}^{n}{Z_{i,t}U_{i,t}\vu_i^T\vx_{i,t}y_{i,t}}\\
&=\half \vu_i^T\paren{I+\sum_{t=1}^{n}{Z_{i,t}U_{i,t}}\vxiit\vxiit^T}\vu_i-\sum_{t=1}^{n}{Z_{i,t}U_{i,t}\vu_i^T\vx_{i,t}y_{i,t}}\\
&=\half \vu_i^T A_{i,n} \vu_i-\sum_{t=1}^{n}{Z_{i,t}U_{i,t}\vu_i^T\vx_{i,t}y_{i,t}}.
\end{split}
\label{sec_order_1}
\end{equation} 
The vectors $\vu_i$ can be replaced with their scaled version, $c\vu_i$. introducing the trivial inequality, $1-x\le \max\braces{1-x,0}$ we get 
\begin{equation}
\begin{split}
cZ_{i,t}U_{i,t}\paren{1-\vu_i^T\vx_{i,t}y_{i,t}} \le cZ_{i,t}U_{i,t}\max\braces{1-\vu_i^T\vx_{i,t}y_{i,t},0}\\
-cZ_{i,t}U_{i,t}\vu_i^T\vx_{i,t}y_{i,t}\le-cZ_{i,t}U_{i,t}+c \lossp{}\paren{\vu_i^T\vx_{i,t}y_{i,t}}.
\label{sec_order_2}
\end{split}
\end{equation}
Rearranging an plugging \eqref{sec_order_2} and \eqref{sec_order_1} 

\begin{equation}
\begin{split}
\half \sum_{t=1}^{n}Z_{i,t}U_{i,t}&\paren{\hat{p}_{i,t}^2-2y_{i,t}\hat{p}_{i,t}-\frac{r_{i,t}}{1+r_{i,t}}+r_{i,t}\hat{p}_{i,t}^{2}+2c}\\
&\le\frac{c^2}{2} \vu_i^T A_{i,n} \vu_i+c\sum_{t=1}^{n}{Z_{i,t}U_{i,t}\lossp{}\paren{\vu_i^T\vx_{i,t}y_{i,t}}}
\label{sec_order_2}
\end{split}
\end{equation}
Recall that $U_{i,t}=M_{i,t}+G_{i,t}$ we will split the inequality into two different cases. First we will take into consideration the cases when an error update was performed, i.e.$M_{i,t}=1$, in which we have $-y_{i,t}\hat{p}_{i,t}=\abs{\hat{p}_{i,t}}$. In this case we need to consider also two subcases, when $\Theta\paren{\abs{\hat{p}_{i,t}},r_{i,t}}\ge0$   and when $\Theta\paren{\abs{\hat{p}_{i,t}},r_{i,t}}<0$. Beginning with the former subcase, recall that for this case 
$\mathbb{E}_{t-1}\brackets{Z_{i,t}}=\frac{1}{D_{t}}\frac{2c}{2c+\paren{\Theta\paren{\abs{\hat{p}_{j,t}},r_{j,t}}}}$, we get 
\begin{equation*}
\begin{split}
\mathbb{E}&\brackets{Z_{i,t}U_{i,t}\paren{\hat{p}_{i,t}^2-2y_{i,t}\hat{p}_{i,t}-\frac{r_{i,t}}{1+r_{i,t}}+r_{i,t}\hat{p}_{i,t}^{2}+2c}}\\
&=\mathbb{E}\brackets{\mathbb{E}_{t-1}\brackets{Z_{i,t}}U_{i,t}\paren{\hat{p}_{i,t}^2-2y_{i,t}\hat{p}_{i,t}-\frac{r_{i,t}}{1+r_{i,t}}+r_{i,t}\hat{p}_{i,t^{}}^{2}+2c}}\\
&=2c\mathbb{E}\brackets{\frac{1}{D_t}U_{i,t}}.
\end{split}
\end{equation*}
When $\Theta\paren{\abs{\hat{p}_{i,t}},r_{i,t}}<0$  the conditional expectation becomes $\mathbb{E}_{t-1}\brackets{Z_{i,t}}=\frac{1}{D_t}$ and the thus,
\begin{equation*}
\begin{split}
\mathbb{E}&\brackets{Z_{i,t}U_{i,t}\paren{\hat{p}_{i,t}^2-2y_{i,t}\hat{p}_{i,t}-\frac{r_{i,t}}{1+r_{i,t}}+r_{i,t}\hat{p}_{i,t}^{2}+2c}}\\
&=\mathbb{E}\brackets{\mathbb{E}_{t-1}\brackets{Z_{i,t}}U_{i,t}\paren{\hat{p}_{i,t}^2+2\abs{\hat{p}_{i,t}}-\frac{r_{i,t}}{1+r_{i,t}}+r_{i,t}\hat{p}_{i,t}^{2}+2c}}\\
&\ge\mathbb{E}\brackets{\mathbb{E}_{t-1}\brackets{Z_{i,t}}U_{i,t}\paren{-\frac{r_{i,t}}{1+r_{i,t}}+2c}}\\
&\ge 2c \mathbb{E}\brackets{\frac{1}{D_t}U_{i,t}}-\frac{r_{i,t}}{1+r_{i,t}}\mathbb{E}\brackets{\frac{1}{D_t}}.
\end{split}
\end{equation*}
Now we examine the case where an update was performed, but there was no mistake. In this case, $0\le y_{i,t}\hat{p}_{i,t}$ and the aggressive update was performed. Recall the bound on the margin for such case and using \eqref{sec_order_2} ,
we bound the margin as follows\begin{equation*}
0\le y_{i,t}\hat{p}_{i,t}\le \theta({r_{i,t}})=\frac{-1+\sqrt{1+r_{i,t}}}{1+r_{i,t}}.
\end{equation*}
We can bound now,
\begin{equation*}
\begin{split}
\hat{p}_{i,t}^2&-2y_{i,t}\hat{p}_{i,t}-\frac{r_{i,t}}{1+r_{i,t}}+r_{i,t}\hat{p}_{i,t}^{2}+2c\\
&=(1+r_{i,t})\hat{p}_{i,t}^2-2y_{i,t}\hat{p}_{i,t}+\frac{r_{i,t}}{1+r_{i,t}}-2\frac{r_{i,t}}{1+r_{i,t}}+2c\\
&=f(y_{i,t}\hat{p}_{i,t})-2\frac{r_{i,t}}{1+r_{i,t}}+2c
\end{split}
\end{equation*}
\end{proof}          
where $f(y_{i,t}\hat{p}_{i,t})=(1+r_{i,t})\hat{p}_{i,t}^2-2y_{i,t}\hat{p}_{i,t}+\frac{r_{i,t}}{1+r_{i,t}}$ is a quadratic convex function with two non-negative roots $\frac{1\pm\sqrt{1-r_{i,t}}}{1+r_{i,t}}$ and we know that the margin is lower than the smaller root , $y_{i,t}\hat{p}_{i,t}\le\frac{1-\sqrt{1-r_{i,t}}}{1+r_{i,t}}$ which leads to the inequality $f(y_{i,t}\hat{p}_{i,t})\ge0$ so we bound 
\begin{equation*}
\begin{split}
\mathbb{E}&\brackets{Z_{i,t}U_{i,t}\paren{\hat{p}_{i,t}^2-2y_{i,t}\hat{p}_{i,t}-\frac{r_{i,t}}{1+r_{i,t}}+r_{i,t}\hat{p}_{i,t}^{2}+2c}}\\
&=\mathbb{E}\brackets{\mathbb{E}_{t-1}\brackets{Z_{i,t}}U_{i,t}\paren{f(y_{i,t}\hat{p}_{i,t})-2\frac{r_{i,t}}{1+r_{i,t}}+2c}}\\
&\ge\mathbb{E}\brackets{\mathbb{E}_{t-1}\brackets{Z_{i,t}}U_{i,t}\paren{-2\frac{r_{i,t}}{1+r_{i,t}}+2c}}\\
&\ge 2c\mathbb{E}\brackets{\frac{1}{D_t}U_{i,t}}-\frac{2r_{i,t}}{1+r_{i,t}}\mathbb{E}\brackets{\frac{1}{D_t}\frac{2r_{i,t}}{1+r_{i,t}}}.
\end{split}
\end{equation*}
Summarize the results we get,
\begin{equation}
\begin{split}
\half \sum_{t=1}^{n}&\mathbb{E}\brackets{Z_{i,t}U_{i,t}\paren{\hat{p}_{i,t}^2-2y_{i,t}\hat{p}_{i,t}-\frac{r_{i,t}}{1+r_{i,t}}+r_{i,t}\hat{p}_{i,t}^{2}+2c}}\\
&\ge c \sum_{t\in\mathcal{M}}\mathbb{E}\brackets{\frac{1}{D_t}U_{i,t}}+c \sum_{t\in\mathcal{G}}\mathbb{E}\brackets{\frac{1}{D_t}U_{i,t}}\\
&-\half\sum_{t\in\mathcal{A\cap M}}\frac{r_{i,t}}{1+r_{i,t}}\mathbb{E}\brackets{\frac{1}{D_t}}
-\sum_{t\in\mathcal{A\cap G}}\frac{r_{i,t}}{1+r_{i,t}}\mathbb{E}\brackets{\frac{1}{D_t}}
\end{split}
\end{equation}
Combining the result of the last inequality with the expectation of the  \eqref{sec_order_2}, recall that  $\sum_{t\in\mathcal{M}}U_{i,t}=M_i$ , and  $\sum_{t\in\mathcal{G}}U_{i,t}=G_i$ we get,   
\begin{equation*}
\begin{split}
c \sum_{t\in\mathcal{M}}\mathbb{E}\brackets{\frac{1}{D_t}U_{i,t}}+c \sum_{t\in\mathcal{G}}\mathbb{E}&\brackets{\frac{1}{D_t}U_{i,t}}
-\frac{1}{2c}\sum_{t\in\mathcal{A\cap M}}\frac{r_{i,t}}{1+r_{i,t}}\mathbb{E}\brackets{\frac{1}{D_t}}\\
-\frac{1}{c}\sum_{t\in\mathcal{A\cap U}}\frac{r_{i,t}}{1+r_{i,t}}\mathbb{E}\brackets{\frac{1}{D_t}}
\le& \frac{c}{2} \vu_i^T \mathbb{E}\brackets{A_{i,n}} \vu_i+\sum_{t=1}^{n}\mathbb{E}\brackets{{Z_{i,t}U_{i,t}\lossp{}\paren{\vu_i^T\vx_{i,t}y_{i,t}}}}
\end{split}
\end{equation*}

The normalization can also been bound by
\begin{equation*}
\begin{split}
D_{t}=2c\sum_{i=1}^{K}{\left({2c+\paren{\Theta\paren{\abs{\hat{p}_{i,t}},r_{i,t}}}_+}\right)^{-1}}& \le  2c\sum_{m=1}^{K}{\frac{1}{2c}}=K
\end{split}
\end{equation*}
which leads to


\begin{equation*}
\begin{split}
\mathbb{E}\brackets{M_{i}}+ &\mathbb{E}\brackets{G_{i}}
-\frac{1}{2c}\mathbb{E}\sum_{t\in\mathcal{A\cap M}}\brackets{\frac{r_{i,t}}{1+r_{i,t}}}
-\frac{1}{c}\mathbb{E}\sum_{t\in\mathcal{A\cap U}}\brackets{\frac{r_{i,t}}{1+r_{i,t}}}\\
\le& \frac{Kc}{2} \vu_i^T \mathbb{E}\brackets{A_{i,n}} \vu_i+K\sum_{t=1}^{n}\mathbb{E}\brackets{{Z_{i,t}U_{i,t}\lossp{}\paren{\vu_i^T\vx_{i,t}y_{i,t}}}}
\end{split}
\end{equation*}



\begin{equation*}
\begin{split}
\mathbb{E}\brackets{M_{i}}\le&\frac{Kc}{2} \vu_i^T \mathbb{E}\brackets{A_{i,n}} \vu_i+K\sum_{t=1}^{n}\mathbb{E}\brackets{{Z_{i,t}U_{i,t}\lossp{}\paren{\vu_i^T\vx_{i,t}y_{i,t}}}} -\mathbb{E}\brackets{G_{i}}\\
&+\frac{1}{2c}\mathbb{E}\sum_{t\in\mathcal{A\cap M}}\brackets{\frac{r_{i,t}}{1+r_{i,t}}}
+\frac{1}{c}\mathbb{E}\sum_{t\in\mathcal{A\cap U}}\brackets{\frac{r_{i,t}}{1+r_{i,t}}}\\
\le& \frac{Kc}{2} \vu_i^T \mathbb{E}\brackets{A_{i,n}} \vu_i+K\mathbb{E}\brackets{\sum_{t=1}^{n}{Z_{i,t}U_{i,t}\lossp{}\paren{\vu_i^T\vx_{i,t}y_{i,t}}}}-\mathbb{E}\brackets{G_{i}}\\
&+\frac{1}{c}\sum_{t\in\mathcal{A}}\brackets{\frac{r_{i,t}}{1+r_{i,t}}}
\end{split}
\end{equation*}
Now, we can summarize the inequality over all of the tasks,
\begin{equation*}
\begin{split}
\mathbb{E}\brackets{M}\le &\frac{cK}{2}\sum_{i=1}^{K}  \vu_i^T \mathbb{E}\brackets{A_{i,n}} \vu_i+K\mathbb{E}\brackets{\sum_{i=1}^{K}\sum_{t=1}^{n}Z_{i,t}U_{i,t}{\lossp{}\paren{\vu_i^T\vx_{i,t}y_{i,t}}}}-\mathbb{E}\brackets{G}\\
&+\frac{1}{c}\mathbb{E}\sum_{i=1}^{K}\sum_{t\in\mathcal{A}}\brackets{\frac{r_{i,t}}{1+r_{i,t}}}
\end{split}
\end{equation*}
 
\begin{equation*}
\begin{split}
\mathbb{E}\brackets{M}\le &\frac{cK}{2}\sum_{i=1}^{K}  \vu_i^T \mathbb{E}\brackets{A_{i,n}} \vu_i+K{\bar L}_{1,n}-\mathbb{E}\brackets{G}\\
&+\frac{1}{c}\mathbb{E}\sum_{i=1}^{K}\sum_{t\in\mathcal{A}}\brackets{\frac{r_{i,t}}{1+r_{i,t}}}
\end{split}
\end{equation*} 



