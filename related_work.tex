\chapter{Related Work}

In the past few years there is a large volume of work on multi-task learning, which clearly we can not 
cover here. The reader is referred to a recent survey on the topic~\cite{10.1109/TKDE.2009.191}. 
Most of this work is focused on exploring relations between tasks, i.e. %that is, 
find similarities dissimilarities  between tasks, and use it to share data directly ,e.g. 
\cite{NIPS2012_0706}, or model parameters as in ~\cite{Evgeniou:2004:RML:1014052.1014067,Daume:2010:FES:1870526.1870534,DBLP:journals/ml/ArgyriouEP08}. 
In the online settings there are only a handful of work on multi-task learning. 
~\cite{DBLP:conf/colt/DekelLS06} consider the setting where all algorithms are evaluated using a 
global loss function, and all work towards the shared goal of minimizing it. 
~\cite{DBLP:conf/colt/LugosiPS09}, assume that there are constraints on the predictions of all 
learners, and focus in the expert setting. \cite{Agarwal:EECS-2008-138}, formalize the 
problem in the framework of stochastic convex programming with few matrix regularization, each 
captures some assumption about the relation between the models. 
~\cite{DBLP:journals/jmlr/CavallantiCG10} and ~\cite{cesa2006incremental}, 
assume a known relation between tasks which is exploited during learning. 
Unlike these approaches, we assume the ability to share an annotator rather than data or parameters, 
thus our methods can be applied to problems with no common input space.

Our analysis is derived from ~\cite{cesa2006worst}, yet they focus in selective 
sampling (see also \cite{cesa2009robust,dekel2010robust}), that is, making individual binary decisions of 
whether to query, while our algorithm always query, and needs to decide for which task to query on each 
round.
%
Finally, there have been recent work in contextual bandits,~\cite{kakade2008efficient,
hazan2011newtron,DBLP:journals/ml/CrammerG13}, each with slightly different assumptions. 
To the best of our knowledge, we are the first to consider decoupled exploration and exploitation in this 
context. Finally, there is recent work in learning with relative or preference feedback in various 
settings~\cite{DBLP:conf/colt/YueBKJ09,DBLP:journals/jcss/YueBKJ12,DBLP:conf/icml/YueJ11,DBLP:journals/corr/abs-1111-0712}. 
Unlike this work, our work allows again decoupled exploitation and exploration, 
and also non-relevant feedback.

\fbox{Check if the rest below is relevant}
In the other hand, there are also works that have been done on binary classification with partial feedback . 
Selective sampling algorithms such as first and second order selective sampling perceptron and   
BBQ  of Cesa-Bianci et al~\cite{cesa2006worst,cesa2009robust} deals with binary selective 
sampling algorithms. Dekel et al ~\cite{dekel2010robust} have also a multiple teachers selective sampling 
algorithm, whit setting that is similar to SHAMPO with $\kappa$ updates per round , but is force all of the 
inputs to be in the same space.

~\cite{dietterich1995solving} introduced the use of ECOC matrix to reduce a 
multi-class classification problem into a multi-task problem we can use this method, combines with 
SHAMPO to get multiclass classification. Algorithms like contextual bandits such the Banditron of 
~\cite{kakade2008efficient} and Newtron of ~\cite{hazan2011newtron}  used the 
exploration exploitation method in order to reduce the regret, but in their setting, for each round the choice 
is also the prediction, so it suitable to our setting only for the case of One-vs-Rest.
